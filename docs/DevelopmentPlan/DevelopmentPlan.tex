\documentclass{article}

\usepackage{booktabs}
\usepackage{tabularx}

\title{Development Plan\\\progname}

\author{\authname}

\date{}

\input{../Comments}
%% Common Parts

\newcommand{\progname}{Chest Scan} % PUT YOUR PROGRAM NAME HERE
\newcommand{\authname}{Team 16, Ace
\\ Harrison Chiu
\\ Hamza Issa
\\ Ahmad Hamadi
\\ Jared Paul
\\ Gurnoor Bal} % AUTHOR NAMES                  

\usepackage{hyperref}
    \hypersetup{colorlinks=true, linkcolor=blue, citecolor=blue, filecolor=blue,
                urlcolor=blue, unicode=false}
    \urlstyle{same}
                                


\begin{document}

\maketitle

\begin{table}[hp]
\caption{Revision History} \label{TblRevisionHistory}
\begin{tabularx}{\textwidth}{llX}
\toprule
\textbf{Date} & \textbf{Developer(s)} & \textbf{Change}\\
\midrule
Date1 & Name(s) & Description of changes\\
23/09/23 & Ahmad Hamadi & Completed proof of concept demonstration plan\\
23/09/24 & Ahmad Hamadi & Completed Appexdix\\
... & ... & ...\\
\bottomrule
\end{tabularx}
\end{table}

\newpage{}

\wss{Put your introductory blurb here.  Often the blurb is a brief roadmap of
what is contained in the report.}

\wss{Additional information on the development plan can be found in the
\href{https://gitlab.cas.mcmaster.ca/courses/capstone/-/blob/main/Lectures/L02b_POCAndDevPlan/POCAndDevPlan.pdf?ref_type=heads}
{lecture slides}.}

\section{Confidential Information?}

\wss{State whether your project has confidential information from industry, or
not.  If there is confidential information, point to the agreement you have in
place.}

\wss{For most teams this section will just state that there is no confidential
information to protect.}
\section{IP to Protect}

\wss{State whether there is IP to protect.  If there is, point to the agreement.
All students who are working on a project that requires an IP agreement are also
required to sign the ``Intellectual Property Guide Acknowledgement.''}

\section{Copyright License}

\wss{What copyright license is your team adopting.  Point to the license in your
repo.}

\section{Team Meeting Plan}

\wss{How often will you meet? where?}

\wss{If the meeting is a physical location (not virtual), out of an abundance of
caution for safety reasons you shouldn't put the location online}

\wss{How often will you meet with your industry advisor?  when?  where?}

\wss{Will meetings be virtual?  At least some meetings should likely be
in-person.}

\wss{How will the meetings be structured?  There should be a chair for all meetings.  There should be an agenda for all meetings.}

\section{Team Communication Plan}

\wss{Issues on GitHub should be part of your communication plan.}

\section{Team Member Roles}

\wss{You should identify the types of roles you anticipate, like notetaker,
leader, meeting chair, reviewer.  Assigning specific people to those roles is
not necessary at this stage.  In a student team the role of the individuals will
likely change throughout the year.}

\section{Workflow Plan}

\begin{itemize}
	\item How will you be using git, including branches, pull request, etc.?
	\item How will you be managing issues, including template issues, issue
	classification, etc.?
  \item Use of CI/CD
\end{itemize}

\section{Project Decomposition and Scheduling}

\begin{itemize}
  \item How will you be using GitHub projects?
  \item Include a link to your GitHub project
\end{itemize}

\wss{How will the project be scheduled?  This is the big picture schedule, not
details. You will need to reproduce information that is in the course outline
for deadlines.}

\section{Proof of Concept Demonstration Plan}

The main functionality required for the success of this project is the accurate
analysis of chest X-rays and the generation of corresponding medical reports.
The most important part of this process is ensuring that the convolutional 
neural network (CNN) can correctly identify and extract the key features from
the X-ray images. In addition to recognizing patterns in the images, it is
also important that the model can generate clear and accurate text that 
describes the findings in a way that medical professionals can understand.
Since this is the core functionality of the project, it represents the biggest
risk we need to address.

To show that this risk can be overcome, the Proof of Concept Demo will involve
manually setting up the backend pipeline for the image analysis and report generation.
We will input a set of chest X-rays and run them through the different components of
the model to show that it can correctly analyze the images and produce useful
descriptions of the findings. The goal of the demo is to demonstrate that the system
can accurately process chest X-rays and generate meaningful outputs that could be
used in a medical setting.

\section{Expected Technology}

\wss{What programming language or languages do you expect to use?  What external
libraries?  What frameworks?  What technologies.  Are there major components of
the implementation that you expect you will implement, despite the existence of
libraries that provide the required functionality.  For projects with machine
learning, will you use pre-trained models, or be training your own model?  }

\wss{The implementation decisions can, and likely will, change over the course
of the project.  The initial documentation should be written in an abstract way;
it should be agnostic of the implementation choices, unless the implementation
choices are project constraints.  However, recording our initial thoughts on
implementation helps understand the challenge level and feasibility of a
project.  It may also help with early identification of areas where project
members will need to augment their training.}

Topics to discuss include the following:

\begin{itemize}
\item Specific programming language
\item Specific libraries
\item Pre-trained models
\item Specific linter tool (if appropriate)
\item Specific unit testing framework
\item Investigation of code coverage measuring tools
\item Specific plans for Continuous Integration (CI), or an explanation that CI
  is not being done
\item Specific performance measuring tools (like Valgrind), if
  appropriate
\item Tools you will likely be using?
\end{itemize}

\wss{git, GitHub and GitHub projects should be part of your technology.}

\section{Coding Standard}

\wss{What coding standard will you adopt?}

\newpage{}

\section*{Appendix --- Reflection}

\input{../Reflection.tex}

\begin{enumerate}
    \item Why is it important to create a development plan prior to starting the
    project?\\
    A development plan is crucial because it helps the team outline the project’s
    goals, timeline, and key milestones. It provides a structured roadmap that
    ensures everyone is aligned on the objectives, minimizes risks, and helps
    to identify any potential bottlenecks early. Additionally, it allows the team
    to allocate resources efficiently and track progress, which is vital for staying
    on schedule and meeting deliverables.

    \item In your opinion, what are the advantages and disadvantages of using
    CI/CD?\\
    The main advantage of using Continuous Integration/Continuous Deployment (CI/CD)
    is that it automates the testing and deployment process, leading to faster and 
    more reliable releases. CI/CD ensures that code is regularly integrated, tested,
    and deployed, which reduces the risk of errors and makes it easier to catch bugs early. 
    The disadvantage, however, is that setting up CI/CD pipelines can be complex and 
    time-consuming, especially for projects that are not well-structured from the start. 
    There can also be significant resource overhead when maintaining and updating the pipelines

    \item What disagreements did your group have in this deliverable, if any,
    and how did you resolve them?\\
    Our group had a minor disagreement on the prioritization of certain features during 
    the initial development phase. Some members felt that we should focus more on the 
    user interface, while others believed the core functionality should be developed first.
    We resolved this by conducting a quick team discussion and agreeing to develop the core
    features first, followed by an iterative approach to improving the interface as we go along.
     This allowed us to address both concerns in a balanced manner.
\end{enumerate}

\newpage{}

\section*{Appendix --- Team Charter}

\wss{borrows from
\href{https://engineering.up.edu/industry_partnerships/files/team-charter.pdf}
{University of Portland Team Charter}}

\subsection*{External Goals}

Our team’s external goals for this project include gaining valuable experience to talk
about in job interviews, aiming for an A+ grade, and showcasing the project at the Capstone
EXPO with hopes of receiving recognition or awards.

\subsection*{Attendance}

\subsubsection*{Expectations}

The team expects all members to attend meetings on time and stay until the meeting is over
unless prior notice is given. Leaving early or missing meetings is discouraged unless 
necessary and communicated ahead of time.

\subsubsection*{Acceptable Excuse}

Acceptable excuses for missing a meeting include personal emergencies, illness, or prior
commitments that were communicated beforehand. Unacceptable excuses include general
forgetfulness or poor time management without prior notice.

\subsubsection*{In Case of Emergency}

In the case of an emergency, the team member should notify the group through our communication
platform (e.g. Discord or email) as soon as possible. They are also expected to provide an update
on their progress and reassign any pending work if necessary.

\subsection*{Accountability and Teamwork}

\subsubsection*{Quality} 

The team expects members to come to meetings prepared with relevant updates and complete tasks
to a high standard. Deliverables should be polished and completed on time to avoid delays in
project progress.

\subsubsection*{Attitude}

Team members are expected to maintain a positive and collaborative attitude. Everyone’s ideas
should be respected, and constructive feedback is encouraged. We will adopt a code of conduct
to ensure professional and courteous interactions. In case of conflicts, the team will resolve
issues through open communication or bring in a mediator if needed.

\subsubsection*{Stay on Track}

The team will use regular progress check-ins and project management tools (GitHub)
to ensure everyone is contributing as expected. Members who perform well will be recognized during
meetings. If someone consistently underperforms, we will discuss the issue as a team and consider
assigning lighter tasks or escalating to the TA if needed. Failure to meet targets may lead to
consequences, such as taking on extra work or contributing to team rewards like bringing snacks to meetings.

\subsubsection*{Team Building}

To build team cohesion, we plan to have occasional informal gatherings, such as virtual coffee breaks
or in-person meetups, depending on availability. These activities will help us bond and maintain a 
positive team dynamic.

\subsubsection*{Decision Making} 

Our team will make decisions through a consensus-based approach, ensuring all voices are heard. If
consensus cannot be reached, we will vote. In case of disagreements, we will hold a separate discussion
to explore all viewpoints and come to a solution that works for the majority.

\end{document}
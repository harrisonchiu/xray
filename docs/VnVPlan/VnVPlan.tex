\documentclass[12pt, titlepage]{article}

\usepackage{parskip}
\usepackage{float}
\usepackage{graphicx}
\usepackage{longtable}
\usepackage{booktabs}
\usepackage{tabularx}
\usepackage{hyperref}
\hypersetup{
  colorlinks,
  citecolor=blue,
  filecolor=black,
  linkcolor=red,
  urlcolor=blue
}
\usepackage[margin=1in,footskip=0.25in]{geometry}
\usepackage[round]{natbib}

%% Comments

\usepackage{color}

\newif\ifcomments\commentstrue %displays comments
%\newif\ifcomments\commentsfalse %so that comments do not display

\ifcomments
\newcommand{\authornote}[3]{\textcolor{#1}{[#3 ---#2]}}
\newcommand{\todo}[1]{\textcolor{red}{[TODO: #1]}}
\else
\newcommand{\authornote}[3]{}
\newcommand{\todo}[1]{}
\fi

\newcommand{\wss}[1]{\authornote{blue}{SS}{#1}} 
\newcommand{\plt}[1]{\authornote{magenta}{TPLT}{#1}} %For explanation of the template
\newcommand{\an}[1]{\authornote{cyan}{Author}{#1}}

%% Common Parts

\newcommand{\progname}{Scanalyze AI} % PUT YOUR PROGRAM NAME HERE
\newcommand{\authname}{Team 16, Ace
\\ Hamza Issa
\\ Ahmad Hamadi
\\ Jared Paul
\\ Gurnoor Bal} % AUTHOR NAMES                  

\usepackage{hyperref}
    \hypersetup{colorlinks=true, linkcolor=blue, citecolor=blue, filecolor=blue,
                urlcolor=blue, unicode=false}
    \urlstyle{same}
                                


\begin{document}

\title{System Verification and Validation Plan for \progname{}}
\author{\authname}
\date{\today}

\maketitle

\pagenumbering{roman}

\section*{Revision History}

\begin{tabularx}{\textwidth}{p{6cm}p{2cm}X}
  \toprule {\bf Date} & {\bf Author} & {\bf Notes}\\
  \midrule
  4 November 2024 & Harrison & Section 4 \\
  4 November 2024 & Hamza & Section 5 \\
  4 November 2024 & Gurnoor & Section 1-3 \\
  4 November 2024 & Jared & Section 3, 5 \\
  4 November 2024 & Ahmad & Section 5 \\
  \bottomrule
\end{tabularx}

~\\
% \wss{The intention of the VnV plan is to increase confidence in the software.
%   However, this does not mean listing every verification and validation technique
%   that has ever been devised.  The VnV plan should also be a \textbf{feasible}
%   plan. Execution of the plan should be possible with the time and team available.
%   If the full plan cannot be completed during the time available, it can either be
%   modified to ``fake it'', or a better solution is to add a section describing
% what work has been completed and what work is still planned for the future.}

% \wss{The VnV plan is typically started after the requirements stage, but before
%   the design stage.  This means that the sections related to unit testing cannot
%   initially be completed.  The sections will be filled in after the design stage
%   is complete.  the final version of the VnV plan should have all sections filled
% in.}

\newpage

\tableofcontents

\listoftables

\newpage

\section{Symbols, Abbreviations, and Acronyms}

Table 1, includes the definitions and descriptions of all relevant symbols,
abbreviations and acronyms used in this VnV Plan document.

\begin{longtable}[c]{|p{0.3\textwidth}|p{0.7\textwidth}|}
  \hline
  \textbf{Symbol, Abbreviation or Acronym} & \textbf{Definiton or Description} \\ \hline
  \textbf{ML} & Machine Learning: A branch of artificial intelligence that involves the use of algorithms to allow computers to learn from and make predictions based on data. This is a core technology used in the project for analyzing chest X-rays. \\ \hline
  \textbf{DL} & Deep Learning: A subset of machine learning involving neural networks with many layers, used to analyze various types of data, including images. \\ \hline
  \textbf{DICOM} & Digital Imaging and Communications in Medicine: A standard for transmitting, storing, and sharing medical imaging information. It is used to manage medical images in the proposed solution. \\ \hline
  \textbf{CNN} & Convolutional Neural Network: A type of deep learning model specifically designed for processing structured grid data like images, used in the project for chest X-ray analysis. \\ \hline
  \textbf{EHR} & Electronic Health Record: A digital version of a patient's paper chart, used for storing patient information and history that may be integrated with the proposed solution. \\ \hline
  \textbf{API} & Application Programming Interface: A set of rules and protocols for building and interacting with software applications, enabling the integration of the proposed solution with other systems. \\ \hline
  \textbf{MC} & Mandated Constraints: Various constraints placed on the project’s proposed solution that must be adhered to throughout the development process. \\ \hline
  \textbf{FR} & Functional Requirement: A requirement that specifies what functionality the project’s proposed solution must provide to meet user needs. \\ \hline
  \textbf{NFR} & Nonfunctional Requirement: A requirement that specifies criteria that can be used to judge the operation of a system, rather than specific behaviors (e.g., performance, usability). \\ \hline
  \textbf{BUC} & Business Use Case: A scenario that describes how the proposed solution can be used within a business context to achieve specific goals. \\ \hline
  \textbf{PUC} & Product Use Case: A scenario that details how an individual user will interact with the proposed solution to achieve specific tasks. \\ \hline
  \textbf{MVP} & Minimum Viable Product: A version of the proposed solution that includes only the essential features required to meet the core needs of the users and stakeholders. \\ \hline
  \textbf{MG} & Module Guide \\ \hline
  \textbf{MIS} & Module Interface Specification \\ \hline
  \textbf{PoC} & Proof of Concept \\ \hline
  \textbf{SRS} & Software Requirements Specification \\ \hline
  \textbf{FRTC} & Functional Requirements Test Case \\ \hline
  \textbf{NFRTC} & Nonfunctional Requirements Test Case \\ \hline
  \textbf{VnV} & Verification and Validation \\ \hline
\end{longtable}

\pagebreak
\newpage

\pagenumbering{arabic}

This document outlines the verification and validation (VnV) plan for the project's proposed solution. It begins with a summary of the software's general functions, the objectives of the VnV plan (including in-scope and out-of-scope elements), and references relevant documentation.

It lists the VnV team members and describes the process for verifying the SRS, design, VnV plan, and implementation, including the automated testing tools used. The VnV roadmap starts with verifying the SRS, followed by the design, VnV plan, and implementation. After these steps, the software will undergo validation, with milestones achieved through automated tools and guidance from the project supervisor. It also shows the tests done (system and unit tests) to each major module to verify its functionality and output.

\section{General Information}
\textbf{Current Content:} Describes verification and validation approach for AI-assisted chest X-ray diagnostics. \\
\textbf{\textcolor{blue}{Update Needed:}} Shift focus to verifying \textbf{multi-class classification accuracy}, ensuring robustness against \textbf{dataset biases, model performance issues, and classification reliability} across multiple disease categories.

\subsection{Summary}
The software being developed is the "Chest Scan" AI diagnostic tool, designed to support radiologists by automating chest X-ray analysis and generating diagnostic reports. This tool will leverage convolutional neural networks (CNNs) to detect and classify lung and heart conditions from X-ray images, thus enhancing diagnostic efficiency and accuracy. The system aims to reduce radiologist workload, improve diagnostic precision, and ultimately elevate patient care. Key functions include image processing, disease identification, and report generation accessible via a user-friendly web interface.

\subsection{Objectives}
The V\&V Plan for the "Chest Scan" diagnostic tool is designed with three core objectives that ensure the tool achieves its intended clinical accuracy, secures sensitive patient information, and provides a practical, usable interface for healthcare professionals. Each objective addresses a critical aspect of the tool's functionality and clinical readiness, from the AI model’s diagnostic performance to the protection of data and ease of use. These objectives guide the verification and validation steps necessary to support confidence in the system’s effectiveness within healthcare environments.

\begin{enumerate}
  \item AI Model Accuracy and Diagnostic Reliability
    \begin{itemize}
      \item Verify that the model can accurately analyze chest X-ray images, identifying targeted conditions such as pneumonia and cardiomegaly with sufficient reliability.
      \item Minimize false negatives and positives to ensure diagnostic reliability, helping radiologists make better-informed decisions.
      \item Validate that the model produces clinically useful outputs, including structured findings and location-based annotations on X-ray images, to streamline interpretation by radiologists.
    \end{itemize}
  \item Data Security and Access Control
    \begin{itemize}
      \item Confirm that patient data is stored and accessed securely, meeting healthcare industry data protection requirements.
      \item Ensure that the system supports secure authentication, granting access only to authorized users while preventing unauthorized access.
      \item Validate compliance with privacy standards (such as HIPAA), focusing on data encryption, secure storage practices, and access restrictions for sensitive information.
    \end{itemize}
  \item Usability and System Accessibility
    \begin{itemize}
      \item Ensure that the software interface is user-friendly, enabling radiologists to log in, access diagnostic results, and interact with generated reports intuitively.
      \item Verify that the web-based interface is efficient, visually accessible, and provides an organized layout, which supports radiologists in reviewing diagnostic information quickly and accurately.
      \item Check that the system integrates smoothly with medical imaging infrastructure (e.g., PACS) and supports data transfer between adjacent systems in clinical environments.
    \end{itemize}
\end{enumerate}

\subsubsection{Out of Scope for the V\&V Plan}
Given the project’s timeframe and resource limitations, the following aspects will not be included in this V\&V Plan. While they may be desirable in a fully deployed product, these areas are either lower priority compared to the core objectives or outside the feasible scope for this development phase. The team has identified these exclusions to maintain focus on critical system functionality, while acknowledging areas that may require future exploration or enhancement.
\begin{itemize}
  \item Industry-Standard UI/UX Optimization: Advanced user interface refinement is deprioritized relative to AI accuracy and data security objectives.
  \item External Library Verification: This plan will not validate the reliability of third-party libraries used, assuming their integrity based on their widespread clinical and research use.
  \item Full Scalability Testing: This MVP will support single-user diagnostic testing, and rigorous multi-user load testing will be out of scope.
\end{itemize}

\subsection{Challenge Level and Extras}
This project is classified at an advanced challenge level due to its dual focus on high-stakes medical diagnostics and complex AI model development, both of which require rigorous standards for accuracy, security, and usability. The challenge lies not only in developing a convolutional neural network (CNN) capable of accurately analyzing chest X-rays but also in ensuring that the software integrates seamlessly within clinical workflows, safeguarding patient privacy, and delivering reliable results under various conditions. Achieving these goals requires the team to balance research and development in AI with a strong emphasis on usability, data security, and healthcare compliance.

The complexity also arises from the project’s need to synthesize academic research with practical implementation, making it a hybrid between a research-oriented and a product-driven development effort. The system must be robust enough to serve as a proof-of-concept for healthcare settings, meeting accuracy benchmarks while maintaining a secure and accessible interface for radiologists.

\subsubsection{Extras}
To enhance the tool’s functionality and make it more user-friendly for clinical adoption, the following extras are planned if resources allow:
\begin{itemize}
  \item Usability Testing with Radiologists: To confirm that the user interface is intuitive, simple usability tests with practicing radiologists will be conducted. This will provide valuable feedback on real-world application and ease of use, ensuring that the tool aligns with the expectations and needs of healthcare professionals.
  \item Research Paper and Documentation: Comprehensive documentation will accompany the MVP, including a research paper summarizing the project’s development, methodologies, and findings. This paper will be valuable for other researchers, detailing insights into the model’s design, training process, and performance metrics. Additionally, this research will be shared with the academic and medical communities, supporting further exploration of AI applications in radiology.
\end{itemize}

These extras, though secondary to the primary objectives, will enhance the tool’s viability and user experience while contributing to the broader knowledge base on AI-driven diagnostics in healthcare.

\subsection{Relevant Documentation}

A set of comprehensive documents will guide and support the verification and validation of the "Chest Scan" diagnostic tool, detailing its requirements, modular structure, and interface specifications. These documents serve as essential references for understanding the system’s architecture and ensuring that all aspects of the tool are accurately validated and verified.
\begin{itemize}
  \item \textbf{Software Requirements Specification (SRS)}
    The SRS outlines both functional and nonfunctional requirements for the tool, setting the foundation for development and testing. It details core features, such as image analysis, disease classification, report generation, and security measures, along with performance criteria and usability goals. The SRS will also include traceability matrices to ensure that all requirements are covered within the testing framework, aligning with the system’s core objectives.
  \item \textbf{Problem Statement Document}
    This document defines the problem the diagnostic tool seeks to address, including the high volume of chest X-rays and the need for efficient diagnostic support for radiologists. It also describes the project's primary goals, stakeholders, and scope, which will help ensure that the V\&V process aligns with the tool's intended purpose and expected clinical impact.
  \item \textbf{Module Guide (MG)}
    The MG will provide a structured breakdown of the tool’s software modules, detailing the hierarchy and interdependencies within the system. It will follow principles of modular decomposition to define each module’s purpose, services, and role within the larger system. By organizing modules based on anticipated changes, this document supports future scalability and ease of maintenance. The MG will also include traceability to ensure that each module meets its corresponding requirements from the SRS, enhancing the effectiveness of the validation process.
  \item \textbf{Module Interface Specification (MIS)}
    The MIS will define the precise interfaces for each module outlined in the MG, including exported constants, access programs, and the semantics of each module’s functionality. It will also specify assumptions, environment variables, and potential state transitions where relevant. This document is critical for the V\&V process, as it ensures that each module can interact correctly with others, conforming to the system’s overall design and requirements.
\end{itemize}

Together, these documents will facilitate a comprehensive understanding of the tool’s architecture and requirements, aiding in the development of thorough and precise V\&V processes to achieve a reliable and secure diagnostic solution.

\section{Plan}

\textbf{Current Content:} Outlines the V\&V methodology for structured report generation. \\
\\
\textbf{\textcolor{blue}{Update Needed:}} Modify to validate \textbf{multi-class classification outputs}, including \textbf{precision, recall, and confusion matrices} for each disease category.


In this section, we will lay out the roadmap for ensuring each part of the project meets the intended standards through verification and validation. Responsibilities are assigned to team members for each task within the V\&V process. This plan will address how we will confirm that the project’s requirements, design documents, V\&V plan, and final implementation align with the specified criteria. Tools and automated testing solutions that support our verification efforts will also be outlined. Additionally, a detailed process for validating the software’s functionality and performance will be included. Our verification tasks will be presented in a structured sequence, moving from requirements analysis to design inspection, implementation testing, and review of the V\&V process itself. Requirements verification will continue throughout development, with design checks occurring as the code is produced. Validation of the implementation will be an ongoing effort, eventually leading to a proof-of-concept phase and final project demonstrations.

\subsection{Verification and Validation Team}

\textbf{Current Content:} Lists team members responsible for V\&V tasks. \\
\textbf{\textcolor{blue}{Update Needed:}} Ensure team expertise includes \textbf{AI model performance assessment, dataset bias evaluation, and clinical validation of classification outputs}.

\begin{table}[h]
  \centering
  \begin{tabular}{|c|c|}
    \hline
    \textbf{
      \begin{tabular}[c]{@{}c@{}}VnV Team\\ Member Name
    \end{tabular}} & \textbf{Summary of Role(s)} \\ \hline
    Amirhossein Sabour &
    \begin{tabular}[c]{@{}c@{}}Advisor, primary reviewer of documentation, a\\ contributor to validation of documentation and code,\\ provide suggestions and corrections of the software\\ and its functionality
    \end{tabular} \\ \hline
    \begin{tabular}[c]{@{}c@{}}Other Design\\ Teams
    \end{tabular} &
    \begin{tabular}[c]{@{}c@{}}Peer reviewers, raise issues and provide feedback/\\ suggestions for documentation improvements
    \end{tabular} \\ \hline
    Gurnoor Bal &
    \begin{tabular}[c]{@{}c@{}}Review other team members’ work to maintain high\\ standards, provide suggestions for improvements,\\ maintain feedback checklists for each work item.
    \end{tabular} \\ \hline
    Jared Paul &
    \begin{tabular}[c]{@{}c@{}}Review other team members’ work to maintain high\\ standards, provide suggestions for improvements,\\ maintain feedback checklists for each work item.
    \end{tabular} \\ \hline
    Ahmad Hamadi &
    \begin{tabular}[c]{@{}c@{}}Review other team members’ work to maintain high\\ standards, provide suggestions for improvements,\\ maintain feedback checklists for each work item.
    \end{tabular} \\ \hline
    Hamza Issa &
    \begin{tabular}[c]{@{}c@{}}Review other team members’ work to maintain high\\ standards, provide suggestions for improvements,\\ maintain feedback checklists for each work item.
    \end{tabular} \\ \hline
    Harrison Chiu &
    \begin{tabular}[c]{@{}c@{}}Review other team members’ work to maintain high\\ standards, provide suggestions for improvements,\\ maintain feedback checklists for each work item.
    \end{tabular} \\ \hline
  \end{tabular}
\end{table}
\pagebreak

\subsection{SRS Verification Plan}
\textbf{Current Content:} Verifies general AI-assisted diagnostic functionality. \\
\textbf{\textcolor{blue}{Update Needed:}} Add checks for \textbf{multi-class disease classification performance}, covering \textbf{false positives/negatives, model interpretability using Grad-CAM, and confidence score validation}.

The SRS Verification Plan for the "Chest Scan" diagnostic tool ensures that all specified requirements are complete, accurate, and verifiable. This section outlines the approach to confirm that the SRS document meets quality standards, providing a solid foundation for subsequent design, implementation, and testing phases.

\subsubsection{Objectives}
The objectives of this SRS Verification Plan are to:
\begin{enumerate}
  \item \textbf{Ensure Requirement Completeness:} Confirm that all essential functionalities, such as X-ray image processing, condition classification, diagnostic report generation, and user interface features, are fully specified in the SRS.
  \item \textbf{Validate Requirement Clarity:} Verify that each requirement is unambiguous, consistent, and comprehensible for both the development and testing teams, reducing the risk of misinterpretation.
  \item \textbf{Check Requirement Feasibility:} Assess each requirement to confirm that it is achievable within the scope and technical constraints of the project.
  \item \textbf{Verify Traceability:} Ensure that each requirement has a unique identifier and can be traced through design and testing phases, facilitating streamlined verification and validation.
\end{enumerate}

\subsubsection{Verification Checklist}
To comprehensively verify the SRS, the following key elements will be checked:
\begin{enumerate}
  \item \textbf{Functional Requirements}
    \begin{itemize}
      \item Clear definitions for each core function, such as:
      \item Processing and analysis of chest X-ray images.
      \item Identification and classification of specific conditions (e.g., pneumonia, cardiomegaly).
      \item Structured diagnostic report generation with condition summaries and annotations.
      \item Specifications for data input/output formats and requirements for public dataset integration (e.g., CheXpert, MIMIC).
    \end{itemize}
  \item \textbf{Nonfunctional Requirements}
    \begin{itemize}
      \item Performance: Defined thresholds for processing speed, response times, and accuracy of diagnoses.
      \item Security: Compliance requirements for patient data protection, including secure storage, access controls, and adherence to privacy standards (e.g., HIPAA).
      \item Usability: User interface design criteria, accessibility considerations, and ease-of-use requirements for radiologists and medical technologists.
      \item Reliability: Expected system uptime and fault-tolerance levels to ensure system stability in clinical environments.
    \end{itemize}
  \item \textbf{Constraints}
    \begin{itemize}
      \item Verification that all system constraints, such as hardware compatibility, software dependencies (e.g., TensorFlow, PyTorch), and budget limitations, are clearly specified.
      \item Identification of external systems the tool must integrate with, like PACS or other hospital information systems.
    \end{itemize}
  \item \textbf{Traceability and Identification}
    \begin{itemize}
      \item Each requirement in the SRS has a unique identifier for easy tracking.
      \item Requirements are linked to corresponding modules (as per the Module Guide) and future test cases, supporting consistency across the V\&V process.
    \end{itemize}
  \item \textbf{Glossary and Terminology}
    \begin{itemize}
      \item Verification that all relevant medical and technical terminology, acronyms, and abbreviations are defined for clarity, preventing misinterpretation.
    \end{itemize}
  \item \textbf{Assumptions and Dependencies}
    \begin{itemize}
      \item Clarity on assumptions (e.g., availability of high-quality datasets, necessary hardware resources).
      \item Dependencies on external datasets, third-party libraries, or existing healthcare infrastructure are explicitly noted.
    \end{itemize}
\end{enumerate}

\subsubsection{Methods}
The following methods will be used to verify the SRS:
\begin{itemize}
  \item \textbf{Requirement Review Meetings:} Conduct collaborative review sessions with stakeholders, including radiologists, AI researchers, and development team members, to verify that all specified requirements align with user needs and project objectives.
  \item \textbf{Traceability Matrix Construction:} Create a traceability matrix linking each requirement to corresponding test cases, ensuring that all requirements can be tested effectively in later stages.
  \item \textbf{Cross-Referencing with MG and MIS:} Check the consistency of requirements with the Module Guide (MG) and Module Interface Specification (MIS) to ensure that each SRS item is appropriately decomposed into modules and interfaces.
\end{itemize}

\subsubsection{Verification Criteria}
The SRS will be considered verified if:
\begin{itemize}
  \item All functional and nonfunctional requirements are clearly stated, necessary for the system’s objectives, and achievable.
  \item Each requirement is supported by a corresponding entry in the traceability matrix.
  \item Stakeholder review feedback confirms that the requirements are complete and aligned with real-world diagnostic and security needs.
\end{itemize}
By following this SRS Verification Plan, we aim to establish a robust foundation for the "Chest Scan" tool, ensuring that every feature aligns with clinical and operational standards while providing clear, testable specifications for development.

\subsection{Design Verification Plan}

\textbf{Current Content:} Ensures modular design of AI-based diagnostic system. \\
\textbf{\textcolor{blue}{Update Needed:}} Verify \textbf{CNN model architecture}, ensuring correct \textbf{layer configurations, activation functions, and feature extraction processes} for multi-class classification.

This Design Verification Plan focuses on verifying that the system’s modular structure and interface specifications, as outlined in the Module Guide (MG) and Module Interface Specification (MIS), align with the requirements in the SRS. Each module’s functionality, integration, and adaptability to potential changes will be reviewed to ensure the system is robust, maintainable, and ready for clinical use.

\subsubsection{Objectives}
The objectives of this Design Verification Plan are to:
\begin{enumerate}
  \item Verify Modular Consistency with Requirements: Ensure that each module in the MG fulfills its corresponding SRS requirements.
  \item Validate Information Hiding and Encapsulation: Confirm that modules are designed to encapsulate specific “secrets,” promoting independence and ease of maintenance.
  \item Check Interface Compatibility: Verify that module interfaces in the MIS support accurate data exchange and align with design constraints.
  \item Assess Anticipated Changes and Traceability: Ensure that anticipated changes are contained within specific modules, preventing system-wide impact and maintaining traceability.
\end{enumerate}

\subsubsection{Verification Checklist}
The following checklist covers key design aspects to be verified based on the MG and MIS documents:
\begin{enumerate}
  \item Module Decomposition
    \begin{itemize}
      \item Modules follow the information-hiding principle, addressing functional and nonfunctional requirements.
      \item Modules are organized by hardware-hiding, behavior-hiding, and software decision layers to support system structure.
      \item Requirements in the SRS are clearly mapped to modules, as shown in the traceability matrix.
    \end{itemize}
  \item Module Interfaces (MIS)
    \begin{itemize}
      \item Interfaces include well-defined input and output formats to ensure seamless integration.
      \item Exported constants, access routines, and state variables are specified for each module’s functionality.
      \item Interface compatibility is confirmed to support consistent interactions without conflicts.
    \end{itemize}
  \item Anticipated and Unlikely Changes
    \begin{itemize}
      \item Design isolates anticipated changes within specific modules, reducing potential impact across the system.
      \item Unlikely changes are documented to avoid unnecessary complexity and improve system simplicity.
    \end{itemize}
  \item Traceability and Use Hierarchy
    \begin{itemize}
      \item The traceability matrix links design elements to SRS requirements, supporting thorough validation.
      \item Use hierarchy between modules is verified to ensure logical dependency, with higher-level modules relying on lower-level modules without cyclic dependencies.
    \end{itemize}
  \item User Interface Design (if applicable)
    \begin{itemize}
      \item For user interaction modules, verify that interface elements are intuitive and accessible.
      \item Ensure compatibility of UI elements with any defined communication protocols, as outlined in the MG.
    \end{itemize}
\end{enumerate}

\subsubsection{Methods}
The following methods will be used to verify the design:
\begin{itemize}
  \item Design Walkthroughs: Conduct walkthroughs with stakeholders and designers to confirm that the modular structure aligns with project goals.
  \item Consistency Checks with SRS: Cross-reference modules in the MG with the SRS requirements to confirm alignment.
  \item Interface Testing: Initial tests on module interfaces based on the MIS to confirm data and function accessibility across modules.
  \item Traceability Matrix Review: Verify that all modules and interfaces link back to requirements, ensuring comprehensive validation.
\end{itemize}

\subsubsection{Verification Criteria}
The design is considered verified if:
\begin{itemize}
  \item Each module aligns with the requirements in the SRS, validated through walkthroughs and interface testing.
  \item The modular structure isolates changes effectively, supporting maintainability.
  \item The traceability matrix confirms all requirements are mapped to design elements.
\end{itemize}
This plan ensures the system’s design is ready for development, with flexibility and reliability suited to clinical applications.

\subsection{Verification and Validation Plan Verification Plan}

This Verification and Validation (V\&V) Plan Verification Plan outlines the process to ensure that the V\&V document itself is thorough, accurate, and effectively designed to support all project goals. By verifying this V\&V document, we aim to confirm that all sections—objectives, scope, methods, and criteria—are clearly defined and cover the requirements, design, and implementation needs outlined in the SRS, MG, and MIS documents.

\subsubsection{Objectives}
The objectives of this V\&V Plan Verification Plan are to:
\begin{enumerate}
  \item Ensure Completeness of Verification and Validation Processes: Verify that this V\&V Plan includes all necessary sections to thoroughly validate the tool’s requirements, design, and implementation.
  \item Confirm Clarity and Traceability: Ensure that each section is clearly written, unambiguous, and easy to follow for all project stakeholders, with traceability to project documents.
  \item Validate Feasibility of Methods: Check that the methods and criteria specified in the V\&V Plan are feasible given project resources and timelines.
  \item Identify Gaps and Overlaps: Ensure that all elements of the system are addressed without unnecessary repetition, minimizing gaps or redundant validations.
\end{enumerate}

\subsubsection{Verification Checklist}
The following checklist identifies key elements in the V\&V Plan to verify for completeness and accuracy:
\begin{enumerate}
  \item Objectives and Scope
    \begin{itemize}
      \item Verify that the V\&V Plan includes well-defined objectives and scope covering all essential elements (SRS, MG, MIS).
      \item Confirm that the plan covers functional and nonfunctional requirements, with a clear boundary on what is out of scope.
      \item Verification and Validation Methods
      \item Ensure that each section (e.g., SRS Verification Plan, Design Verification Plan) includes clearly defined methods and criteria for verifying requirements, design, and implementation.
      \item Confirm that the V\&V methods align with the constraints and objectives laid out in the project’s SRS.
    \end{itemize}
  \item Traceability and Consistency
    \begin{itemize}
      \item Check that the V\&V Plan is traceable to all primary project documents (SRS, MG, MIS) with appropriate references, ensuring alignment and consistency across documents.
      \item Ensure that each V\&V Plan section is traceable back to specific requirements or design aspects from the SRS and MG.
    \end{itemize}
  \item Feasibility and Practicality of Methods
    \begin{itemize}
      \item Confirm that each verification and validation method in the V\&V Plan is achievable within the project’s resources, timeline, and technical constraints.
      \item Check that tools, data, and stakeholder inputs needed for each V\&V process are specified and accessible.
    \end{itemize}
  \item Criteria for Success
    \begin{itemize}
      \item Ensure that the V\&V Plan defines clear criteria for determining the success of each verification and validation step.
      \item Verify that these criteria are realistic and aligned with project requirements, design goals, and intended system functionality.
    \end{itemize}
\end{enumerate}

\subsubsection{Methods}
To verify the V\&V Plan:
\begin{itemize}
  \item Document Review Meetings: Conduct meetings with project stakeholders and team members to review each section of the V\&V Plan for clarity, feasibility, and traceability.
  \item Cross-Referencing with Project Documents: Check each verification and validation method against the SRS, MG, and MIS to confirm alignment and completeness.
  \item Feasibility Assessment: Evaluate the methods and tools proposed in the V\&V Plan to ensure they are practical and achievable with available resources.
\end{itemize}

\subsubsection{Verification Criteria}
The V\&V Plan will be considered verified if:
\begin{itemize}
  \item All sections are complete, clearly written, and aligned with project documents (SRS, MG, MIS).
  \item The verification and validation methods are feasible and cover all essential elements.
  \item Stakeholders confirm that the V\&V Plan is practical and provides a robust approach for testing and validating the tool.
\end{itemize}
This Verification Plan ensures that the V\&V Plan itself is reliable, thorough, and ready to support a comprehensive validation process for the "Chest Scan" diagnostic tool.

\subsection{Implementation Verification Plan}

\textbf{Current Content:} Focuses on AI system implementation correctness. \\
\textbf{\textcolor{blue}{Update Needed:}} Include validation steps for \textbf{multi-class model training stability, bias reduction techniques, and real-world dataset adaptation}.

This Implementation Verification Plan outlines the steps to confirm that the "Chest Scan" diagnostic tool’s final implementation accurately reflects its design, fulfills all specified requirements, and operates as intended. This plan ensures that each implemented feature undergoes rigorous testing to verify functionality, security, and usability as defined in the SRS, MG, and MIS documents.

\subsubsection{Objectives}
The objectives of the Implementation Verification Plan are to:
\begin{itemize}
  \item Verify Functional Accuracy: Ensure that each implemented feature, such as disease detection and report generation, performs correctly according to the requirements.
  \item Confirm Compliance with Security and Usability Standards: Verify that patient data handling, access controls, and user interface meet security and usability guidelines.
  \item Check Integration of Modules: Validate that modules work cohesively and interface accurately with one another, as defined in the MG and MIS.
\end{itemize}

\subsubsection{Verification Methods}
The following methods will be used:
\begin{itemize}
  \item Unit Testing: Test each module independently for correct functionality, ensuring it meets individual requirements from the SRS.
  \item Integration Testing: Test interactions between modules to confirm they work together as specified in the MIS, particularly for data handling and image processing.
  \item System Testing: Evaluate the full system under realistic conditions to verify end-to-end functionality, security, and usability.
\end{itemize}

\subsubsection{Verification Criteria}
The implementation will be considered verified if:
\begin{itemize}
  \item All unit and integration tests pass according to defined requirements.
  \item Security and usability standards are met in system testing.
  \item Stakeholders confirm that the system functions accurately and meets project goals.
\end{itemize}
This plan ensures the implementation is verified to support accurate, secure, and user-friendly operations for clinical use.

\subsection{Automated Testing and Verification Tools}

\textbf{Current Content:} Lists software tools for verification testing. \\
\textbf{\textcolor{blue}{Update Needed:}} Integrate \textbf{automated evaluation scripts} to track \textbf{classification accuracy, fairness metrics, and adversarial robustness testing}.

Automated testing will ensure consistent verification of the "Chest Scan" diagnostic tool’s functionality, accuracy, and security throughout its development. Based on the development plan, a selection of tools will be used to optimize testing efficiency, maintain code standards, and support continuous integration.

\subsubsection{Objectives}
The goals of automated testing are to:
\begin{itemize}
  \item Enhance Testing Efficiency: Use automated tools to streamline testing processes, minimizing manual oversight.
  \item Support Continuous Integration and Deployment (CI/CD): Integrate tests within a CI pipeline to verify code on each update, reducing the risk of regressions.
\end{itemize}

\subsubsection{Tools and Methods}
Based on the development plan, the following tools will be employed:
\begin{itemize}
  \item Unit Testing:
    \begin{itemize}
      \item Pytest: Automates unit tests to ensure each Python module performs correctly, useful for validating core functionalities of the neural network.
      \item Unittest: Alternative for straightforward unit tests, with easy integration into CI workflows.
    \end{itemize}
  \item UI and Integration Testing:
    \begin{itemize}
      \item Selenium: For automated testing of the web interface, checking usability, functionality, and responsiveness.
      \item React Testing Library: For component-level testing within the React frontend, ensuring reliable user interactions.
    \end{itemize}
  \item Continuous Integration:
    \begin{itemize}
      \item GitHub Actions: Handles CI/CD pipeline setup, running tests on each pull request to verify code standards, formatting, and passing tests before merges.
      \item Flake8, Black, and ESLint: Linting and formatting tools integrated with CI to enforce style standards and catch code errors early.
    \end{itemize}
\end{itemize}

\subsubsection{Success Criteria}
Automated testing will be deemed effective if:
\begin{itemize}
  \item CI pipeline tests pass consistently, ensuring code quality on each update.
  \item Functional, integration, and regression tests run smoothly with each deployment, supporting a reliable development workflow.
\end{itemize}
This automated approach ensures a high standard of quality, reducing manual testing needs and enabling quick identification of issues throughout the development lifecycle.

\subsection{Software Validation Plan}
This Software Validation Plan outlines the approach to confirm that the tool meets defined requirements and aligns with stakeholder expectations. External datasets, such as those from MIMIC-CXR and CheXpert, will be utilized to validate model accuracy and functionality. Additionally, user testing sessions with radiologists will help validate usability and practical applicability within clinical workflows.

\subsubsection{Objectives}
The objectives of the Software Validation Plan are to:
\begin{itemize}
  \item Validate Requirement Alignment: Ensure that all requirements defined in the SRS are fulfilled through task-based inspections and stakeholder review sessions.
  \item Gather Stakeholder Feedback: Use review sessions, including the Rev 0 demonstration, as an opportunity to gather feedback from the supervisor and other stakeholders to refine and improve the tool.
  \item Confirm Real-World Usability: Validate that the tool’s performance and interface meet clinical usability standards through task-based testing with radiologists.
\end{itemize}

\subsubsection{Validation Methods}
The following methods will be employed to confirm software effectiveness:
\begin{itemize}
  \item Task-Based Inspections: Test key functionalities using real-world tasks that simulate common diagnostic processes to ensure accurate performance.
  \item Review and Demo Sessions: Conduct Rev 0 and Rev 1 demonstrations with the supervisor to validate the implementation against the requirements and incorporate feedback for improvement.
  \item User Testing: Gather feedback from radiologists during user testing sessions, focusing on ease of use, accuracy, and workflow compatibility.
\end{itemize}

\subsubsection{Validation Criteria}
The software will be considered validated if:
\begin{itemize}
  \item External datasets confirm the model’s diagnostic accuracy against real-world X-ray data.
  \item Review and demo session feedback indicates alignment with requirements and stakeholder expectations.
  \item User testing results show that radiologists find the tool effective and user-friendly in a clinical setting.
\end{itemize}
This validation plan leverages both external data and stakeholder input to ensure the tool is reliable, accurate, and clinically relevant.

\section{System Tests}

\textbf{Current Content:} Describes test cases for verifying system functionality. \\
\textbf{\textcolor{blue}{Update Needed:}} Expand test cases to include:
\begin{itemize}
    \item \textbf{Classification accuracy validation} for each disease category.
    \item \textbf{Handling of ambiguous cases} where multiple diseases are present.
    \item \textbf{Robustness testing} for poor-quality X-ray images.
\end{itemize}

In this subsection, the system test cases are categorized based on different areas of
functionality to ensure a thorough verification process. The following subsets of test cases are
designed to validate the handling of input data, processing and disease signature generation,
disease identification and classification, report generation, display functionality, user
interface access, data storage and security, and authentication and authorization mechanisms.
These subsets cover the key functionalities described in the functional requirements necessary for
the success of this project.

For each test case, references to the relevant functional requirements from the SRS that are
covered by it are included in the test case derivation part.

\subsection{Tests for Functional Requirements}

\textbf{Current Content:} General tests for AI-assisted reporting. \\
\textbf{\textcolor{blue}{Update Needed:}} Focus on verifying \textbf{multi-class classification performance}, ensuring \textbf{balanced accuracy across disease types}.

\subsubsection{Handling Input Data}
\textbf{FRTC1}\\
\textbf{Title:} Chest X-ray Image Input Acceptance\\
\textbf{Control:} Manual\\
\textbf{Initial State:} The system is in a stable state with all components initialized and ready to receive input.\\
\textbf{Input:} A sample chest X-ray image in a valid format (e.g., DICOM, JPEG, PNG).\\
\textbf{Output:} The system accepts and reads the chest x-ray image successfully, with no error messages or system anomalies.\\
\textbf{Test Case Derivation:} The expected output is justified based on FR1 (The system shall accept and read chest x-ray images as input).\\
\textbf{How the test will be performed:}
\begin{enumerate}
  \item Manually upload a sample chest X-ray image in a valid format through the user interface.
  \item Observe the system's response to the input, checking for any error messages or unexpected behavior.
  \item Verify that the system successfully accepts and reads the chest x-ray image.
\end{enumerate}
\vspace{1em}

\textbf{FRTC2}\\
\textbf{Title:} Invalid chest x-ray Image Format Rejection\\
\textbf{Control:} Manual\\
\textbf{Initial State:} The system is in a stable state with all components initialized and ready to receive input.\\
\textbf{Input:} A sample image in an invalid format (e.g., .txt, .docx).\\
\textbf{Output:} The system rejects the invalid image input, accompanied by an appropriate error message.\\
\textbf{Test Case Derivation:}
\begin{enumerate}
  \item Manually attempt to upload an image in an invalid format through the user interface.
  \item Observe the system's response to the input, checking for any error messages or unexpected behavior.
  \item Verify that the system rejects the invalid image and displays an appropriate error message.
\end{enumerate}
\vspace{1em}

\subsubsection{Processing and Disease Signature Generation}
\textbf{FRTC3}\\
\textbf{Title:} Disease Signature Generation Using Diffusion Model\\
\textbf{Control:} Automatic\\
\textbf{Initial State:} The system is ready with the diffusion model loaded and operational.\\
\textbf{Input:} A sample chest x-ray image and specified disease locations (if applicable).\\
\textbf{Output:} The system processes the chest x-ray image using the diffusion model and generates an output image with added disease signatures at specified locations.\\
\textbf{Test Case Derivation:}
\begin{enumerate}
  \item Submit a sample chest x-ray image with specified disease locations to the system.
  \item Allow the system to process the image using the diffusion model.
  \item Verify that the output image has disease signatures generated at the specified locations.
  \item Compare the output with expected results to ensure correctness.
\end{enumerate}
\vspace{1em}

\textbf{FRTC4}\\
\textbf{Title:} Diffusion Model Error Handling with Invalid Input\\
\textbf{Control:} Automatic\\
\textbf{Initial State:} The system is ready with the diffusion model loaded.\\
\textbf{Input:} An invalid or corrupted chest x-ray image.\\
\textbf{Output:} The system handles the error gracefully, providing an appropriate error message without crashing.\\
\textbf{Test Case Derivation:}
\begin{enumerate}
  \item Submit an invalid or corrupted chest x-ray image to the system.
  \item Observe the system's response.
  \item Verify that the system detects the invalid input and provides an appropriate error message without crashing.
\end{enumerate}
\vspace{1em}

\subsubsection{Disease Identification and Classification}
\textbf{FRTC5}\\
\textbf{Title:} Correct Disease Classification on chest x-ray Image\\
\textbf{Control:} Automatic (with manual verification)\\
\textbf{Initial State:} The system is in a stable state with the disease identification model loaded and ready.\\
\textbf{Input:} A chest x-ray image containing known disease patterns for Pneumonia, Atelectasis, Cardiomegaly, and Pleural Effusion.\\
\textbf{Output:} The system identifies and classifies the diseases present in the image with an accuracy greater than 90\% for each disease.\\
\textbf{Test Case Derivation:}
\begin{enumerate}
  \item Input a set of chest x-ray images with known diagnoses into the system.
  \item Allow the system to process the images and identify diseases.
  \item Compare the system's output with the known diagnoses.
  \item Calculate the accuracy for each disease classification.
  \item Verify that the accuracy meets or exceeds 90\% for each disease.
\end{enumerate}
\vspace{1em}

\textbf{FRTC6}\\
\textbf{Title:} Disease Absence Correct Identification\\
\textbf{Control:} Automatic (with manual verification)\\
\textbf{Initial State:} The system is in a stable state with the disease identification model loaded.\\
\textbf{Input:} A chest x-ray image with no disease present (healthy patient).\\
\textbf{Output:} The system correctly identifies the absence of diseases, with an accuracy greater than 90\%.\\
\textbf{Test Case Derivation:}
\begin{enumerate}
  \item Input a set of normal (healthy) chest x-ray images into the system.
  \item Allow the system to process the images and identify diseases.
  \item Verify that the system reports the absence of diseases accurately.
  \item Calculate the accuracy of disease absence identification.
  \item Confirm that the accuracy is greater than 90\%.
\end{enumerate}
\vspace{1em}

\subsubsection{Report Generation}
\textbf{FRTC7}\\
\textbf{Title:} Structured Diagnostic Report Generation\\
\textbf{Control:} Automatic (with manual verification)\\
\textbf{Initial State:} The system has successfully identified diseases in the input chest x-ray image.\\
\textbf{Input:} chest x-ray images with known disease findings.\\
\textbf{Output:} The system generates a diagnostic report detailing the diseases, their severity, and the locations in the image where the abnormalities were detected.\\
\textbf{Test Case Derivation:}
\begin{enumerate}
  \item Input a chest x-ray image with known disease findings into the system.
  \item Allow the system to process the image and generate the diagnostic report.
  \item Review the generated report to ensure it includes:
  \item Detected diseases
  \item Severity levels
  \item Locations of abnormalities in the image
  \item Compare the report to the known findings to verify correctness.
\end{enumerate}
\vspace{1em}

\textbf{FRTC8}\\
\textbf{Title:} Report Generation Error Handling\\
\textbf{Control:} Automatic\\
\textbf{Initial State:} The system is in a stable state but with incomplete disease identification results (simulate a processing error).\\
\textbf{Input:} A scenario where disease identification results are incomplete or missing.\\
\textbf{Output:} The system handles the incomplete data gracefully, indicating in the report that certain findings could not be determined.\\
\textbf{Test Case Derivation:}
\begin{enumerate}
  \item Simulate a scenario where the disease identification module fails to provide results.
  \item Attempt to generate a diagnostic report.
  \item Observe how the system handles the missing data.
  \item Verify that the report indicates which findings are missing and does not crash or produce misleading information.
\end{enumerate}
\vspace{1em}

\subsubsection{Display of Heatmaps and Reports}
\textbf{FRTC9}\\
\textbf{Title:} Heatmap Display on chest x-ray Images\\
\textbf{Control:} Manual\\
\textbf{Initial State:} The system has successfully identified diseases and generated heatmaps indicating disease locations.\\
\textbf{Input:} A processed chest x-ray image with heatmaps generated.\\
\textbf{Output:} The system overlays heatmaps on the chest x-ray images, accurately highlighting regions where disease signatures are present.\\
\textbf{Test Case Derivation:}
\begin{enumerate}
  \item Access the user interface and retrieve the processed chest x-ray image with heatmaps.
  \item Observe the displayed image with heatmaps overlaid.
  \item Verify that the heatmaps accurately correspond to the locations of detected diseases.
  \item Compare the heatmaps to the original findings to ensure correctness.
\end{enumerate}
\vspace{1em}

\textbf{FRTC10}\\
\textbf{Title:} Diagnostic Report and Heatmap Access via Web Interface\\
\textbf{Control:} Manual\\
\textbf{Initial State:} The system has processed chest x-ray images, generated reports and heatmaps, and stored them.\\
\textbf{Input:} User access through the web-based user interface.\\
\textbf{Output:} The system successfully displays diagnostic reports and heatmaps on the web interface for authorized users.\\
\textbf{Test Case Derivation:}
\begin{enumerate}
  \item Log in to the web-based user interface as an authorized user.
  \item Navigate to the section containing diagnostic reports and heatmaps.
  \item Open a diagnostic report and the associated heatmap.
  \item Verify that the report and heatmap are displayed correctly and are accessible.
\end{enumerate}
\vspace{1em}

\subsubsection{User Interface Access}
\textbf{FRTC11}\\
\textbf{Title:} Authorized User Access to Diagnostic Results\\
\textbf{Control:} Manual\\
\textbf{Initial State:} The system is operational, with diagnostic reports and heatmaps available, and user accounts set up.\\
\textbf{Input:} Authorized user login credentials.\\
\textbf{Output:} The user gains access to the diagnostic reports and heatmaps via the web-based interface.\\
\textbf{Test Case Derivation:}
\begin{enumerate}
  \item Attempt to log in to the web-based user interface using valid credentials of an authorized user.
  \item Upon successful login, navigate to the diagnostic results.
  \item Verify that the user can access and view the diagnostic reports and heatmaps.
\end{enumerate}
\vspace{1em}

\textbf{FRTC12}\\
\textbf{Title:} Unauthorized User Access Denied\\
\textbf{Control:} Manual\\
\textbf{Initial State:} The system is operational with diagnostic results available.\\
\textbf{Input:} Invalid login credentials or an unauthorized user.\\
\textbf{Output:} The system denies access to diagnostic reports and heatmaps, displaying an appropriate error message.\\
\textbf{Test Case Derivation:}
\begin{enumerate}
  \item Attempt to log in using invalid credentials or as a user without the necessary permissions.
  \item Observe the system's response.
  \item Verify that access is denied and an appropriate error message is displayed.
\end{enumerate}
\vspace{1em}

\subsubsection{Data Storage and Security}
\textbf{FRTC13}\\
\textbf{Title:} Secure Storage of Patient Data and Diagnostic Results\\
\textbf{Control:} Manual and Automatic\\
\textbf{Initial State:} The system is operational and has processed chest x-ray images and generated diagnostic reports.\\
\textbf{Input:} chest x-ray images, patient data, and diagnostic reports.\\
\textbf{Output:} The system securely stores and retrieves chest x-ray images, diagnostic reports, and patient information in a secure database.\\
\textbf{Test Case Derivation:}
\begin{enumerate}
  \item Input patient data and chest x-ray images into the system.
  \item Allow the system to process the images and generate diagnostic reports.
  \item Verify that all data is stored in the backend database securely.
  \item Attempt to retrieve the stored data and ensure that data integrity is maintained.
  \item Verify that unauthorized access to the database is prevented.
\end{enumerate}
\vspace{1em}

\textbf{FRTC14}\\
\textbf{Title:} Data Encryption in Storage and Transit\\
\textbf{Control:} Automatic\\
\textbf{Initial State:} The system is operational with data storage and network communication components active.\\
\textbf{Input:} Patient data, chest x-ray images, and diagnostic reports during storage and retrieval operations.\\
\textbf{Output:} The data is encrypted during storage and transit, ensuring patient confidentiality.\\
\textbf{Test Case Derivation:}
\begin{enumerate}
  \item Monitor data storage processes to verify that encryption is applied to data at rest.
  \item Monitor network communication during data transmission to ensure data is encrypted in transit (e.g., using HTTPS).
  \item Use security tools to attempt to intercept or access data without authorization.
  \item Verify that data remains secure and encrypted, preventing unauthorized access.
\end{enumerate}
\vspace{1em}

\subsubsection{Authentication and Authorization}
\textbf{FRTC15}\\
\textbf{Title:} User Authentication with Valid Credentials\\
\textbf{Control:} Manual\\
\textbf{Initial State:} The system is operational with user accounts created.\\
\textbf{Input:} Valid login credentials for an authorized user.\\
\textbf{Output:} The system authenticates the user and grants access according to the user's role.\\
\textbf{Test Case Derivation:}
\begin{enumerate}
  \item Attempt to log in using valid credentials.
  \item Verify that the system authenticates the user successfully.
  \item Verify that the user has access to functions appropriate for their role.
\end{enumerate}
\vspace{1em}

\textbf{FRTC16}\\
\textbf{Title:} User Authentication with Invalid Credentials\\
\textbf{Control:} Manual\\
\textbf{Initial State:} The system is operational.\\
\textbf{Input:} Invalid login credentials.\\
\textbf{Output:} The system denies access and displays an appropriate error message.\\
\textbf{Test Case Derivation:}
\begin{enumerate}
  \item Attempt to log in using invalid credentials (wrong username or password).
  \item Observe the system's response.
  \item Verify that access is denied and an appropriate error message is displayed.
\end{enumerate}
\vspace{1em}

\textbf{FRTC17}\\
\textbf{Title:} Role-Based Access Control Enforcement\\
\textbf{Control:} Manual\\
\textbf{Initial State:} The system is operational with users assigned different roles (e.g., radiologist, administrator, technician).\\
\textbf{Input:} Users attempting to access functions outside their authorized roles.\\
\textbf{Output:} The system restricts actions based on user roles, preventing unauthorized access.\\
\textbf{Test Case Derivation:}
\begin{enumerate}
  \item Log in as a user with a specific role (e.g., technician).
  \item Attempt to perform actions that are restricted to other roles (e.g., accessing patient reports).
  \item Verify that the system denies access to unauthorized functions.
  \item Repeat the test with other roles to ensure proper enforcement of access controls.
\end{enumerate}
\vspace{1em}

\subsection{Tests for Nonfunctional Requirements}

\textbf{Current Content:} Evaluates usability, performance, and security. \\
\textbf{\textcolor{blue}{Update Needed:}} Introduce tests for:
\begin{itemize}
    \item \textbf{Model interpretability} using Grad-CAM heatmaps.
    \item \textbf{Latency measurement} for classification processing times.
    \item \textbf{Bias detection} in training data and classification outputs.
\end{itemize}

In this subsection, the system test cases are designed to validate key non-functional requirements
that are critical to the user experience and overall system performance. The tests focus on the
appearance, usability, performance, and security aspects of the software. Each test case
references the relevant non-functional requirements from the SRS to ensure comprehensive coverage
and traceability.

\subsubsection{Appearance Requirements}
\textbf{NFRTC1}\\
\textbf{Title:} Verification of Calming Color Scheme\\
\textbf{Control:} Manual\\
\textbf{Initial State:} The web application is deployed and accessible via a web browser.\\
\textbf{Input:} Access the web application using a standard web browser.\\
\textbf{Output:} The web app displays a color scheme that combines white and soft tones of blue.\\
\textbf{Test Case Derivation:} The expected output is justified based on NF-AR1 (The web app will incorporate a cool, and calming color scheme that combines white and soft tones of blue).\\
\textbf{How the test will be performed:}
\begin{enumerate}
  \item Open the web application in a web browser.
  \item Observe the color scheme of the interface, noting the primary colors used.
  \item Verify that the colors are white and soft tones of blue.
  \item Conduct a user survey where participants rate whether the color scheme contributes positively to their experience.
\end{enumerate}
\vspace{1em}

\textbf{NFRTC2}\\
\textbf{Title:} Consistency of Simple and Uncluttered Style\\
\textbf{Control:} Manual\\
\textbf{Initial State:} The web application is fully functional with all interface elements loaded.\\
\textbf{Input:} Navigate through different pages and features of the web application.\\
\textbf{Output:} The web app maintains a consistent, simple, and uncluttered style throughout.\\
\textbf{Test Case Derivation:} The expected output is justified based on NF-SR1 and NF-SR2 (The web app will have a consistent simple style; The software interface shall maintain a clean and organized aesthetic).\\
\textbf{How the test will be performed:}
\begin{enumerate}
  \item Access the main page of the web application.
  \item Navigate through all available features and pages.
  \item Observe the layout, design elements, and interface consistency.
  \item Ensure that there are no unnecessary or distracting design elements.
  \item Conduct usability tests where participants navigate the interface and report on ease of navigation and aesthetic appeal.
\end{enumerate}
\vspace{1em}

\subsubsection{Usability and Humanity Requirements}
\textbf{NFRTC3}\\
\textbf{Title:} Ease of Performing Core Functionality Without Guidance\\
\textbf{Control:} Manual\\
\textbf{Initial State:} The web application is accessible, and new users are ready to interact with it.\\
\textbf{Input:} First-time users attempt to generate a chest X-ray image without any prior training or guidance.\\
\textbf{Output:} Users are able to perform basic tasks, like generating an image, without needing extra guidance, and no irrelevant features are present.\\
\textbf{Test Case Derivation:} The expected output is justified based on NF-EUR0 (The web app should focus only on core functionality, with no extra features that might confuse users).\\
\textbf{How the test will be performed:}
\begin{enumerate}
  \item Recruit a group of participants with no prior experience using the web app.
  \item Ask participants to generate a chest X-ray image using the app without providing any instructions.
  \item Observe their interactions, noting any difficulties or confusion.
  \item Verify that users can generate an image without assistance within the first minute of use.
  \item Ensure that no irrelevant features distract or confuse the users during the process.
\end{enumerate}
\vspace{1em}

\textbf{NFRTC4}\\
\textbf{Title:} Intuitiveness and Memorability of the Software\\
\textbf{Control:} Manual\\
\textbf{Initial State:} Users have completed 1-2 training sessions on using the software.\\
\textbf{Input:} One month after training, users attempt to navigate the software and perform basic functions without assistance.\\
\textbf{Output:} Users report they can navigate the software and perform basic functions without assistance.\\
\textbf{Test Case Derivation:} The expected output is justified based on NF-EU2 (The software shall be intuitive enough that users can remember how to operate it after 1-2 training sessions).\\
\textbf{How the test will be performed:}
\begin{enumerate}
  \item Provide 1-2 training sessions to a group of users on how to use the software.
  \item Wait one month without any further interaction with the software.
  \item Ask users to perform basic tasks on the software without any assistance.
  \item Collect feedback through surveys or interviews regarding their ability to navigate and use the software.
  \item Analyze responses to ensure that at least 80\% of users could operate the software effectively without assistance.
\end{enumerate}
\vspace{1em}

\textbf{NFRTC5}\\
\textbf{Title:} Learning Curve Assessment\\
\textbf{Control:} Manual\\
\textbf{Initial State:} New users with no prior experience are ready to use the web application.\\
\textbf{Input:} Users attempt to generate a chest X-ray image upon first use of the app.\\
\textbf{Output:} Users are able to generate an image within the first minute of using the app, even without prior experience.\\
\textbf{Test Case Derivation:} The expected output is justified based on NF-LR0 (The web app should be easy to learn, allowing users to quickly understand how to use it without needing tutorials or training).\\
\textbf{How the test will be performed:}
\begin{enumerate}
  \item Recruit participants with no prior experience using the web app.
  \item Provide them with access to the web application without any instructions or tutorials.
  \item Ask them to generate a chest X-ray image.
  \item Measure the time taken from first accessing the app to successfully generating an image.
  \item Verify that users can accomplish this within one minute.
  \item Collect any feedback on their learning experience.
\end{enumerate}
\vspace{1em}

\textbf{NFRTC6}\\
\textbf{Title:} Realism of Generated Chest X-Ray Images\\
\textbf{Control:} Manual (with expert evaluation)\\
\textbf{Initial State:} The system is operational and capable of generating chest X-ray images.\\
\textbf{Input:} Generate a set of chest X-ray images using the diffusion model.\\
\textbf{Output:} Radiologists or other medical professionals are unable to easily distinguish between the generated X-rays and real-world X-rays after reviewing them.\\
\textbf{Test Case Derivation:} The expected output is justified based on NF-UPR0 (The chest X-ray images generated by the diffusion model should closely resemble real-world X-rays used in hospitals).\\
\textbf{How the test will be performed:}
\begin{enumerate}
  \item Generate a sample set of chest X-ray images using the diffusion model.
  \item Compile a mixed set of images including both generated images and real-world X-rays.
  \item Present this set to a group of medical professionals (e.g., radiologists).
  \item Ask them to identify which images are real and which are generated.
  \item Collect and analyze the results.
  \item Verify that medical professionals cannot easily distinguish between the two sets, indicating high realism in the generated images.
\end{enumerate}
\vspace{1em}

\subsubsection{Performance Requirements}
\textbf{NFRTC7}\\
\textbf{Title:} Image Generation Speed Test\\
\textbf{Control:} Automatic\\
\textbf{Initial State:} The system is operational and ready to process image generation requests.\\
\textbf{Input:} A request to generate a chest X-ray image.\\
\textbf{Output:} The system generates the chest X-ray image within 1 minute of the user request.\\
\textbf{Test Case Derivation:} The expected output is justified based on NF-SLR0 (The system should generate a chest X-ray image within a reasonable amount of time after a request is made).\\
\textbf{How the test will be performed:}
\begin{enumerate}
  \item Initiate an image generation request through the web app.
  \item Start a timer immediately upon submission of the request.
  \item Record the time taken for the system to generate and display the chest X-ray image.
  \item Ensure that the image is generated and available to the user within 1 minute.
  \item Repeat the test multiple times to account for variations in processing time.
\end{enumerate}
\vspace{1em}

\textbf{NFRTC8}\\
\textbf{Title:} System Stability and Uptime Monitoring\\
\textbf{Control:} Automatic (over an extended period)\\
\textbf{Initial State:} The system is deployed and accessible to users.\\
\textbf{Input:} Continuous monitoring of system availability over a specified period (e.g., one month).\\
\textbf{Output:} The system remains available at least 99\% of the time, with minimal downtime (no more than 1 hour of downtime per week).\\
\textbf{Test Case Derivation:} The expected output is justified based on NF-RFTR0 (The system should be stable and available for use most of the time).\\
\textbf{How the test will be performed:}
\begin{enumerate}
  \item Set up monitoring tools to track system uptime and downtime continuously over the testing period.
  \item Record any instances of system downtime, noting the duration and cause.
  \item Calculate the total uptime percentage over the testing period.
  \item Verify that the system maintains at least 99\% uptime.
  \item Ensure that any downtime does not exceed 1 hour per week.
\end{enumerate}
\vspace{1em}

\textbf{NFRTC9}\\
\textbf{Title:} Single Image Generation Capacity Enforcement\\
\textbf{Control:} Manual and Automatic\\
\textbf{Initial State:} The system is operational and processing an image generation request.\\
\textbf{Input:} Attempt to initiate multiple simultaneous image generation requests.\\
\textbf{Output:} The system allows only one image to be generated at a time, rejecting additional simultaneous requests with an appropriate message.\\
\textbf{Test Case Derivation:} The expected output is justified based on NF-CR0 (The system should only generate one chest X-ray image at a time).\\
\textbf{How the test will be performed:}
\begin{enumerate}
  \item Initiate an image generation request and confirm that the process has started.
  \item While the first request is being processed, attempt to initiate additional image generation requests from the same or different user accounts.
  \item Observe the system's response to the additional requests.
  \item Verify that the system prevents additional simultaneous image generation requests and provides an appropriate notification to the user.
  \item Ensure that the first image generation process continues uninterrupted.
\end{enumerate}
\vspace{1em}

\subsubsection{Security Requirements}
\textbf{NFRTC10}\\
\textbf{Title:} Public Accessibility Without Authentication\\
\textbf{Control:} Manual\\
\textbf{Initial State:} The web application is deployed and accessible over the internet.\\
\textbf{Input:} Access the web app from multiple devices and locations without providing any login credentials.\\
\textbf{Output:} The web app is accessible to anyone without needing to sign in or provide any credentials.\\
\textbf{Test Case Derivation:} The expected output is justified based on NF-AR0 (The web app will be publicly accessible to anyone, without the need for login or credentials).\\
\textbf{How the test will be performed:}
\begin{enumerate}
  \item Access the web application from different devices (e.g., desktop, laptop, tablet, smartphone) and different web browsers.
  \item Attempt to use the web app's functionalities without logging in or providing any personal information.
  \item Verify that all core functionalities are accessible without any authentication prompts.
    Ensure consistent access across various devices and locations.
\end{enumerate}
\vspace{1em}

\textbf{NFRTC11}\\
\textbf{Title:} Verification of Non-Collection of Personal Data\\
\textbf{Control:} Manual and Automatic\\
\textbf{Initial State:} The web application is operational and ready for user interaction.\\
\textbf{Input:} Interact with the web app and monitor data collection processes.\\
\textbf{Output:} The web app does not request or store any personal information from users during their interactions with the system.\\
\textbf{Test Case Derivation:} The expected output is justified based on NF-PR0 (The system will not collect any personal data from users interacting with the web app).\\
\textbf{How the test will be performed:}
\begin{enumerate}
  \item Use the web application extensively, attempting to access all available features.
  \item Observe whether the app requests any personal information (e.g., name, email, contact details).
  \item Use network monitoring tools to inspect data being sent to the server during interactions.
  \item Verify that no personal data is being transmitted or stored.
  \item Review the application's privacy policy and any data handling practices to ensure compliance.
\end{enumerate}
\vspace{1em}

\subsubsection{Supportability Requirements}
\textbf{NFRTC12}\\
\textbf{Title:} Availability and Sufficiency of User Documentation\\
\textbf{Control:} Manual\\
\textbf{Initial State:} The web application is operational with user documentation provided.\\
\textbf{Input:} Access the user documentation and attempt to experiment with the diffusion model based on the guidelines.\\
\textbf{Output:} Users can access the documentation and experiment with the model without needing further assistance.\\
\textbf{Test Case Derivation:} The expected output is justified based on NF-SR0 (The system will be self-sufficient, with all necessary documentation provided so that users can understand how to experiment with the diffusion model).\\
\textbf{How the test will be performed:}
\begin{enumerate}
  \item Locate and access the user documentation or guidelines provided within the web app.
  \item Read through the documentation to understand how to use the diffusion model.
  \item Attempt to perform tasks such as generating images and adjusting any available parameters based on the instructions.
  \item Evaluate whether the documentation provides clear and sufficient guidance for using the model.
  \item Collect feedback from users on the clarity and usefulness of the documentation.
\end{enumerate}
\vspace{1em}

\subsubsection{Cultural Requirements}
\textbf{NFRTC13}\\
\textbf{Title:} Functionality of Browser-Based Language Translation\\
\textbf{Control:} Manual\\
\textbf{Initial State:} The web application is operational and written in English.\\
\textbf{Input:} Access the web app using a web browser's translation feature to translate it into another language.\\
\textbf{Output:} The web app is fully functional in English, and users are able to translate it into other languages using their browser if necessary.\\
\textbf{Test Case Derivation:} The expected output is justified based on NF-CR0 (The web app will be usable by people from any background or culture with no restrictions such as in region).\\
\textbf{How the test will be performed:}
\begin{enumerate}
  \item Access the web application using a browser that supports automatic translation (e.g., Google Chrome).
  \item Use the browser's translation feature to translate the web app into a different language.
  \item Navigate through the app and perform core functionalities while in the translated version.
  \item Verify that the translation is applied consistently across the interface.
  \item Ensure that the app remains fully functional after translation, with no broken elements or errors.
\end{enumerate}
\vspace{1em}

\subsection{Untested Non-Functional Requirements}
Some non-functional requirements are not tested due to limitations in scope, feasibility, or time
constraints. These requirements are recognized but are not included in the testing plan.
\begin{itemize}
  \item \textbf{NF-SER0 (Scalability or Extensibility Requirements):} Testing future scalability and updates is beyond our current project scope.
  \item \textbf{NF-LR0 (Longevity Requirements):} We cannot test the system's operation over a span of three years within our testing timeline.
  \item \textbf{NF-MR0 (Maintenance Requirements):} Assessing future maintenance activities isn't feasible during this phase.
  \item \textbf{NF-EPE0 (Expected Physical Environment):} This is an operational expectation and doesn't require specific testing.
  \item \textbf{NF-RIAS0 (Interfacing with Adjacent Systems):} Since the system is self-contained with no external interfaces, no tests are necessary.
  \item \textbf{NF-IR0 (Integrity Requirements):} The system outputs images without additional data protection measures; thus, there's nothing specific to test here.
  \item \textbf{NF-AR0 (Audit Requirements):} No formal security audits are planned, so no related tests are designed.
\end{itemize}

\subsection{Traceability Between Test Cases and Requirements}

\begin{table}[h]
  \centering
  \begin{tabular}{|p{0.7\textwidth}|p{0.23\textwidth}|}
    \hline
    \textbf{Functional Requirement} & \textbf{Test Cases} \\
    \hline
    FR1: The system shall accept and read CXR (Chest X-ray) images as input. & FRTC1, FRTC2 \\
    \hline
    FR2: The system shall process CXR images using a diffusion model to generate disease signatures at specified locations. & FRTC3, FRTC4 \\
    \hline
    FR3: The system shall identify and classify multiple diseases in the CXR image, including (but not limited to) Pneumonia, Atelectasis, Cardiomegaly, and Pleural Effusion. & FRTC5, FRTC6 \\
    \hline
    FR4: The system shall generate a structured diagnostic report from the findings on the CXR image. & FRTC7, FRTC8 \\
    \hline
    FR5: The system shall display heatmaps on the CXR images to indicate the locations of the detected disease signatures. & FRTC9 \\
    \hline
    FR6: The system shall allow users to access the diagnostic reports and heatmaps via a web-based user interface. & FRTC10, FRTC11 \\
    \hline
    FR7: The system shall store patient data, CXR images, and diagnostic reports securely in a backend database. & FRTC13, FRTC14 \\
    \hline
    FR8: The system shall provide authentication and authorization mechanisms to control access to the system. & FRTC11, FRTC12, FRTC15, FRTC16, FRTC17 \\
    \hline
  \end{tabular}
\end{table}

\pagebreak

\begin{table}[h]
  \centering
  \begin{tabular}{|p{0.8\textwidth}|p{0.12\textwidth}|}
    \hline
    \textbf{Non-Functional Requirement} & \textbf{Test Cases} \\ \hline
    NF-AR1: The web app will incorporate a cool, and calming color scheme that combines white and soft tones of blue. & NFRTC1 \\ \hline
    NF-SR1: The web app will have a consistent simple style that makes it easy to use and experiment with the model, without adding unnecessary complexity. & NFRTC2 \\ \hline
    NF-SR2: The software interface shall maintain a clean and organized aesthetic. & NFRTC2 \\ \hline
    NF-EUR0: The web app should focus only on core functionality, with no extra features that might confuse users. & NFRTC3 \\ \hline
    NF-EU2: The software shall be intuitive enough that users can remember how to operate it after 1-2 training sessions. & NFRTC4 \\ \hline
    NF-LR0: The web app should be easy to learn, allowing users to quickly understand how to use it without needing tutorials or training. & NFRTC5 \\ \hline
    NF-UPR0: The chest X-ray images generated by the diffusion model should closely resemble real-world X-rays used in hospitals. & NFRTC6 \\ \hline
    NF-SLR0: The system should generate a chest X-ray image within a reasonable amount of time after a request is made. & NFRTC7 \\ \hline
    NF-RFTR0: The system should be stable and available for use most of the time. & NFRTC8 \\ \hline
    NF-CR0: The system should only generate one chest X-ray image at a time. & NFRTC9, NFRTC13 \\ \hline
    NF-AR0: The web app will be publicly accessible to anyone, without the need for login or credentials. & NFRTC10 \\ \hline
    NF-PR0: The system will not collect any personal data from users interacting with the web app. & NFRTC11 \\ \hline
    NF-SR0: The system will be self-sufficient, with all necessary documentation provided so that users can understand how to experiment with the diffusion model. & NFRTC12 \\ \hline
  \end{tabular}
\end{table}

\section{Unit Test Description}
\textbf{Current Content:} Defines unit test coverage for system components. \\
\textbf{\textcolor{blue}{Update Needed:}} Ensure unit tests cover \textbf{each disease classification module separately}, tracking performance across \textbf{varied dataset conditions}.

This section outlines the approach and rationale for the unit tests designed to verify each module within the chest X-ray diagnostic application. The overall philosophy for test case selection centers on ensuring each module functions correctly in isolation, covering both standard functionality and potential edge cases. This approach provides a robust foundation to identify and address any issues at the unit level before integrating modules into the broader system.

To streamline testing and avoid redundancy, each module’s tests are designed to validate both typical and boundary conditions. This strategy includes creating a baseline test case for normal expected behavior, accompanied by additional edge cases to capture unexpected or extreme inputs. This allows for comprehensive testing without unnecessary detail for every possible scenario.

For efficiency, test cases prioritize high-risk functions and core functionalities. Tests are organized to:
\begin{itemize}
  \item Validate standard functionality with a single test for typical cases.
  \item Test critical edge cases that reflect likely and realistic error conditions (e.g., handling of empty data, invalid formats, and unexpected large data inputs).
  \item Ensure modules perform within expected limits for nonfunctional requirements, such as performance and security, through specialized tests.
\end{itemize}

Where possible, detailed input/output descriptions are omitted in favor of referring to well-documented unit testing code, which maintains readable and meaningful test names. This approach not only saves space but also facilitates future maintenance and expansion, as new edge cases can be added to the codebase directly.

\subsection{Unit Testing Scope}

The scope of unit testing for this project includes all the core modules that are necessary to the chest X-ray diagnostic application. Each core module will be tested independently to verify its functionality (unit tests), handling expected and edge cases.

The primary modules in scope include:
\begin{itemize}
  \item \textbf{ReadXRayImage Module} – Responsible for reading and validating DICOM images
  \item \textbf{DiagnosticAnalysis Module} – Performs AI-based analysis on images to generate diagnostic predictions.
  \item \textbf{ReportBuilder Module} – Compiles the AI-generated results into structured diagnostic reports.
  \item \textbf{DatabaseOperations Module} – Manages patient data storage and retrieval
\end{itemize}

PredictiveModel  Module – Executes the core disease classification logic, supporting accurate and efficient diagnosis.
Modules that are not included in the unit testing scope are:
\begin{itemize}
  \item \textbf{Third-Party Libraries and External APIs} – Any external libraries or APIs used in the system are assumed to have been verified by their respective developers. These components are not unit-tested here but are included in system and integration testing.
\end{itemize}

\subsection{Tests for Functional Requirements}
This section lists unit tests grouped by module, focusing on both standard and edge case scenarios.

\subsubsection{ReadXRayImage Module}
Tests for this module verify that DICOM images are read, validated, and handled correctly.
\begin{enumerate}
  \item \textbf{Test Name:} Retrieve and Validate DICOM Files\\
    \textbf{Type:} Automatic\\
    \textbf{Initial State:} System ready to receive DICOM file\\
    \textbf{Function Tested:} \verb|fetch_and_validate_dicom|\\
    \textbf{Input:} Valid patient data entries from Database\\
    \textbf{Expected Output:} Returns list of valid DICOM files\\
    \textbf{Test Case Derivation:} Confirms that valid directories are read and files retrieved correctly.\\
    \textbf{How the test will be performed:} provide a directory path containing DICOM files; verify that each file is returned as expected.\\
    \textbf{Edge Case:} Empty patient directory or inaccessible directory\\
    \textbf{Expected Output for Edge Case:} Returns an empty list or a specific error indicating no files found.

  \item \textbf{Test Name:}  Check File Format Compatibility\\
    \textbf{Type:}  Automatic\\
    \textbf{Initial State:}  System ready to receive DICOM file\\
    \textbf{Function Tested:}  \verb|is_dicom_format|\\
    \textbf{Input:}  Sample DICOM, JPEG, and corrupted files\\
    \textbf{Expected Output:}  True for DICOM, False for JPEG, Error for corrupted file\\
    \textbf{Test Case Derivation:}  Verifies correct file identification and error handling for incompatible file types.\\
    \textbf{How test will be performed:}  Provide sample files to the function; confirm correct identification or error logging.\\

  \item \textbf{Test Name:}  Handle Corrupted DICOM File Input\\
    \textbf{Type:}  Automatic\\
    \textbf{Initial State:}  System ready to receive DICOM file\\
    \textbf{Function Tested:}  \verb|read_dicom_data|\\
    \textbf{Input:}  Corrupted DICOM file\\
    \textbf{Expected Output:}  Returns an error or logs issue without crashing\\
    \textbf{Test Case Derivation:}  Ensures model handles unreadable data without crashing.\\
    \textbf{How test will be performed:}  Provide a corrupted DICOM file; confirm the model logs an error and skips further processing.\\
    \textbf{Edge Case:}  Partially transferred or incomplete DICOM file\\
    \textbf{Expected Output for Edge Case:}  Gracefully handles error, logs details, and skips further processing\\
\end{enumerate}

\subsubsection{DiagnosticAnalysis Module}
This module generates diagnostic predictions for chest X-rays. It includes tests for typical image inputs, as well as abnormal images.
\begin{enumerate}
  \item \textbf{Test Name:}  Prediction on Standard X-ray Image\\
    \textbf{Type:}  Automatic\\
    \textbf{Initial State:}  Model loaded and ready for analysis\\
    \textbf{Function Tested:}  \verb|analyze_image|\\
    \textbf{Input:}  Raw data for a valid single X-ray image\\
    \textbf{Expected Output:}  Prediction results for any identified conditions.\\
    \textbf{Test Case Derivation:}  Confirms the model’s diagnostic capability with a basic and simple input.\\
    \textbf{How test will be performed:}  Submit a valid X-ray image; verify prediction results match the expected conditions.\\
    \textbf{Edge Case:}  X-ray image that does not portray any health issues\\
    \textbf{Expected Output for Edge Case:}  Processes image as expected and is able to identify there are no health issues in a reasonable processing time.\\

  \item \textbf{Test Name:}  Process Empty Image Data\\
    \textbf{Type:}  Automatic\\
    \textbf{Initial State:}  Model loaded and ready for analysis\\
    \textbf{Function Tested:}  \verb|analyze_image|\\
    \textbf{Input:}  Empty or null image data\\
    \textbf{Expected Output:}  Error message or safe default response\\
    \textbf{Test Case Derivation:}  Confirms that invalid or blank images are appropriately flagged without prediction.\\
    \textbf{How test will be performed:}  Submit empty data; confirm error message or safe response output.\\
    \textbf{Edge Case:}  image with unidentifiable image\\
    \textbf{Expected Output for Edge Case:}  Returns “unreadable image” error\\
\end{enumerate}

\subsubsection{ReportBuilder Module}
This module compiles the prediction results into readable reports. The tests validate accurate report generation and the handling of incomplete input data.
\begin{enumerate}
  \item \textbf{Test Name:}  Generate Summary Report from Prediction Results\\
    \textbf{Type:}  Automatic\\
    \textbf{Function Tested:}  \verb|compile_summary|\\
    \textbf{Initial State:}  Ready to compile report from prediction data\\
    \textbf{Input:}  Prediction results for a single patient (could contain multiple x-ray images)\\
    \textbf{Expected Output:}  Well-structured report summary\\
    \textbf{Test Case Derivation:}  Confirms that valid prediction data generates a complete report.\\
    \textbf{How test will be performed:}  Provide prediction data; verify that the report summary is correctly structured.\\

  \item \textbf{Test Name:}  Handle low quality/quantity Prediction Results\\
    \textbf{Type:}  Automatic\\
    \textbf{Initial State:}  Ready to compile report from prediction data\\
    \textbf{Function Tested:}  \verb|compile_summary|\\
    \textbf{Input:}  Partially complete prediction data\\
    \textbf{Expected Output:}  Complete report with placeholders or indicators for missing data\\
    \textbf{Test Case Derivation:}  Ensures that reports can be generated even when data is partially missing.\\
    \textbf{How test will be performed:}  Provide partial data; verify the report includes placeholders or notes for missing information.\\
    \textbf{Edge Case:}  No predictions available at all\\
    \textbf{Expected Output for Edge Case:}  Outputs a “no findings” report or similar message indicating a lack of results.\\
\end{enumerate}

\subsubsection{DataBaseOperations Module}
This module focuses on storing and retrieving patient data, while ensuring security and data integrity.
\begin{enumerate}
  \item \textbf{Test Name:}  Store Data Attempt\\
    \textbf{Type:}  Automatic\\
    \textbf{Initial State:}  Ready to authenticate user \\
    \textbf{Function Tested:}  \verb|store_patient_data|\\
    \textbf{Input:}  Valid patient data, User Credentials\\
    \textbf{Expected Output:}  Data stored successfully\\
    \textbf{Test Case Derivation:}  Confirms only authorized users can enter data.\\
    \textbf{How test will be performed:}  Submit patient data as an authorized user; verify successful data entry.\\
    \textbf{Edge Case:}  User Credentials are invalid\\
    \textbf{Expected Output for Edge Case:}  Logs warning for invalid user credentials and it timeouts after 5 attempts.\\

  \item \textbf{Test Name:}  Access Data Attempt\\
    \textbf{Type:}  Automatic\\
    \textbf{Initial State:}  Ready to authenticate user\\
    \textbf{Function Tested:}  \verb|retrieve_patient_data|\\
    \textbf{Input:}  User Credentials, valid patient data\\
    \textbf{Expected Output:}  retrieves patient data\\
    \textbf{Test Case Derivation:}  Confirms only authorized users can read data.\\
    \textbf{How test will be performed:}  Read patient data as an authorized user; verify that data was successfully read.\\
    \textbf{Edge Case:}  User Credentials are invalid\\
    \textbf{Expected Output for Edge Case:}  Logs warning for invalid user credentials and it timeouts after 5 attempts.\\
\end{enumerate}

\subsubsection{PredictiveModel Module}
This module is a core component responsible for processing chest X-ray images using advanced AI techniques, including diffusion models with bounding box and inpainting capabilities.
\begin{enumerate}
  \item \textbf{Test Name:}  Diagnose with Inpainting for Object Removal\\
    \textbf{Type:}  Automatic\\
    \textbf{Initial State:}  Model ready for prediction\\
    \textbf{Function Tested:}  \verb|process_and_predict|\\
    \textbf{Input:}  X-ray image containing a visible artifact (e.g., shadow)\\
    \textbf{Expected Output:}  Clear diagnostic results after the artifact is removed and replaced with realistic content.\\
    \textbf{Test Case Derivation:}  Confirms model's ability to enhance image by removing artifacts.\\
    \textbf{How test will be performed:}  provide X-ray with artifact; verify that model removes the artifact and produces an accurate diagnosis.\\
    \textbf{Edge Case:}  Artifact located close to an area of suspected abnormality\\
    \textbf{Expected Output for Edge Case:}  The model correctly removes only the artifact while preserving the surrounding area for correct analysis.\\

  \item \textbf{Test Name:}  Diagnose Condition with Focused Enhancement\\
    \textbf{Type:}  Automatic\\
    \textbf{Initial State:}  Model ready for focused analysis\\
    \textbf{Function Tested:}  \verb|enhance_and_predict|\\
    \textbf{Input:}  X-ray with a faint or hard-to-see abnormal region marked by a bounding box\\
    \textbf{Expected Output:}  Clearer image in the specified area and a correct diagnosis\\
    \textbf{Test Case Derivation:}  Confirms that unfocused areas receive focus- enhancement for better accuracy.\\
    \textbf{How test will be performed:}  Provide X-ray with hard to see areas; confirm that the enhanced area supports accurate diagnosis.\\

  \item \textbf{Test Name:}  Restore Blurred or Damaged Areas in X-ray\\
    \textbf{Type:}  Automatic\\
    \textbf{Initial State:}  Model ready for restoration\\
    \textbf{Function Tested:}  \verb|restore_and_predict|\\
    \textbf{Input:}  X-ray image with blurry or degraded areas\\
    \textbf{Expected Output:}  Restored image with improved clarity and accurate diagnostic results\\
    \textbf{Test Case Derivation:}  Verifies model’s ability to improve clarity in damaged areas.\\
    \textbf{How test will be performed:}  Submit X-ray with blurred regions; confirm that model restores these areas.\\
\end{enumerate}

\subsection{Tests for Nonfunctional Requirements}

Tests in this section ensure that the application meets nonfunctional requirements such as performance and security.

\subsubsection{Performance Testing - PredictiveModel Module}
\begin{enumerate}
  \item \textbf{Test Name:}  Bulk Image Processing Speed\\
    \textbf{Type:}  Automatic\\
    \textbf{Initial State:}  Model loaded and ready for batch processing\\
    \textbf{Input:}  High volume batch of X-ray images\\
    \textbf{Expected Output:}  Processes batch within acceptable time frame.\\
    \textbf{How test will be performed:}  A large batch of X-ray images is submitted to the model for processing. The system’s response time is measured and logged.\\
    \textbf{Edge Case:}  Maximum batch size allowed by the system\\
    \textbf{Expected Output for Edge Case:}  Handles maximum load with minimal delay, logs if processing times exceed threshold.\\
\end{enumerate}

\subsubsection{Security Testing - DataBaseOperations Module}
\begin{enumerate}
  \item \textbf{Test Name:}  Unauthorized Access Block\\
    \textbf{Type:}  Automatic\\
    \textbf{Initial State:}  Ready to Authenticate User\\
    \textbf{Input:}  Access attempt from unauthorized user\\
    \textbf{Expected Output:}  Denies access and logs attempt\\
    \textbf{How test will be performed:}  Send unauthorized access attempts to patient data, verifying the system’s response. The test checks that unauthorized access is blocked.\\
\end{enumerate}

\subsection{Traceability Between Test Cases and Modules}

\begin{table}[h!]
  \centering
  \caption{Legend for Module Labels}
  \begin{tabular}{|c|l|}
    \hline
    \textbf{Module Label} & \textbf{Description} \\
    \hline
    M1 & ReadXRayImage \\
    \hline
    M2 & DiagnosticAnalysis \\
    \hline
    M3 & ReportBuilder \\
    \hline
    M4 & DatabaseOperations \\
    \hline
    M5 & PredictiveModel \\
    \hline
  \end{tabular}
\end{table}

\begin{table}[h!]
  \centering
  \caption{Traceability Matrix for Functional Requirement Test Cases (FRTC)}
  \begin{tabular}{|c|c|c|c|c|c|}
    \hline
    \textbf{FRTC} & \textbf{M1} & \textbf{M2} & \textbf{M3} & \textbf{M4} & \textbf{M5} \\
    \hline
    FRTC1 & X &  &  & X &  \\
    \hline
    FRTC2 &  & X &  &  &  \\
    \hline
    FRTC3 &  & X &  &  & X \\
    \hline
    FRTC4 &  &  & X &  &  \\
    \hline
    FRTC5 &  &  &  & X &  \\
    \hline
    FRTC6 &  & X &  &  & X \\
    \hline
  \end{tabular}
\end{table}

\begin{table}[h!]
  \centering
  \caption{Traceability Matrix for Non-Functional Requirement Test Cases (NFRTC)}
  \begin{tabular}{|c|c|c|c|c|c|}
    \hline
    \textbf{NFRTC} & \textbf{M1} & \textbf{M2} & \textbf{M3} & \textbf{M4} & \textbf{M5} \\
    \hline
    NFRTC1 & X &  &  &  &  \\
    \hline
    NFRTC2 &  & X &  &  & X \\
    \hline
    NFRTC3 &  &  & X &  &  \\
    \hline
    NFRTC4 &  &  &  & X &  \\
    \hline
    NFRTC5 & X &  &  & X &  \\
    \hline
    NFRTC6 &  & X &  &  & X \\
    \hline
    NFRTC7 &  &  &  & X &  \\
    \hline
    NFRTC8 &  & X &  &  & X \\
    \hline
    NFRTC9 &  &  &  &  & X \\
    \hline
    NFRTC10 &  &  & X &  &  \\
    \hline
    NFRTC11 &  &  &  & X &  \\
    \hline
  \end{tabular}
\end{table}

\pagebreak
\newpage

\section{Appendix}
In this section, additional information that complements the V\&V Plan is included.

\subsection{Symbolic Parameters}
The definition of test cases will call for certain symbolic constants. Their values are defined in this section for easy maintenance. See the following table for reference.

\begin{table}[h]
  \centering
  \begin{tabular}{|p{0.3\textwidth}|p{0.7\textwidth}|}
    \hline
    \multicolumn{1}{|c|}{\textbf{Symbolic Name}} & \multicolumn{1}{c|}{\textbf{Definition or Description}} \\ \hline
    DATASET-NAME & The chest X-ray imaging datasets used for development and testing. Details are as follows: \\ \hline
    MIMIC-CXR & A standard labeled chest X-ray dataset commonly used for medical imaging tasks. \\ \hline
    Chest ImaGenome & An extension of MIMIC-CXR, with detailed annotations including 29 bounding boxes on anatomical regions. This dataset links textual report segments to specific bounding boxes, supporting tasks in segmentation and disease localization. \\ \hline
  \end{tabular}
\end{table}

\subsection{Usability Survey Questions?}

The following survey should be filled out by users after using the system for 5 to 15 minutes. This feedback helps evaluate the user experience and identifies areas for improvement.

\subsubsection{User Experience Survey}

\textbf{Time using system:}
Please provide a rating between 0 and 10 for each of the following categories, with 0 being very difficult or undesirable, and 10 being very easy or desirable.
\begin{enumerate}
  \item Ease of Use: 0 1 2 3 4 5 6 7 8 9 10 \\
    (0 = very difficult to use, 10 = very easy to use)
  \item Navigation: 0 1 2 3 4 5 6 7 8 9 10 \\
    (0 = hard to find what you’re looking for, 10 = easy to find what you’re looking for)
  \item Readability: 0 1 2 3 4 5 6 7 8 9 10 \\
    (0 = hard to understand information, 10 = very easy to understand)
  \item Look and Feel: 0 1 2 3 4 5 6 7 8 9 10 \\
    (0 = visually unappealing, 10 = visually appealing)
\end{enumerate}

\subsubsection{Additional Feedback:}
Notes: (Space for users to leave any additional comments or suggestions about the system.)

\newpage{}
\section*{Appendix --- Reflection}

\textbf{Current Content:} Summarizes lessons learned and challenges. \\
\textbf{\textcolor{blue}{Update Needed:}} Expand reflections to cover \textbf{challenges in classification generalization, optimizing hyperparameters, and improving model transparency for clinical adoption}.


\subsection{Question 1}
What went well while writing this deliverable?

\textbf{Harrison Chiu:}  Planning out the modules for functional testing went pretty smoothly. Reviewing SOLID principles helped a lot and I was able to structure each module with a clear purpose, which made creating the tests feel straightforward. I even tried some initial tests on my own machine to catch edge cases early on, and it gave me confidence that I was covering all the important scenarios. That fuzz testing experiment helped me feel like I was on the right track.

\textbf{Hamza Issa:} Working on identifying tests for functional requirements felt really intuitive. After we reached a consensus regarding the functional requirements and our ability to implement it, it was quite simple to devise tests that were complete and meaningful, as part of creating functional requirements is that the expectation action or output is typically discrete.

\textbf{Gurnoor Bal:} Working on the SRS verification was actually pretty satisfying because I got to make sure everything was crystal clear. I found a few areas in the SRS that needed improvement, and being able to fix those early felt good. Plus, setting up the traceability matrix was helpful since it kept things organized and made it easy to link each requirement with its specific test case. It felt like things were clicking into place.

\textbf{Jared Paul:} I liked working on the design verification part because it let me focus on making sure the system was organized and easy to work with. I emphasized modularity and made sure each module was self-contained, which will make the system easier to maintain down the road. I also created a checklist to make sure each module followed solid design practices, so I felt like I was covering the important aspects without missing anything.

\textbf{Ahmad Hamadi:} Writing the validation plan was rewarding because I got to think about how people would actually use the tool, especially radiologists. I set up task-based testing and feedback sessions, so we’ll get a real sense of how practical the tool is. Using actual datasets to check the model’s accuracy felt like a solid choice too. It felt like I was creating a plan that would make sure the tool was both accurate and user-friendly.

\subsection{Question 2}
What pain points did you experience during this deliverable, and how did you resolve them? \\

\textbf{Hamza Issa:} I found that in order to create a V\&V that was really complete meant that I needed to not just need a basic abstract understanding of the theory around Diffusion models, rather I needed to actually understand the theory, how it is typically implemented today in research and industry, so much so that I needed to find out the type of data structure to expect as output which would be a rather complex matrix. This entire process of learning was really challenging as I was a novice in the subject. But I managed to solve this by purchasing a course on Udemy regarding Theory and Implementation of Diffusion models which helped me really better understand and write a V\&V plan that actually made sense in the context of diffusion models.

\textbf{Harrison Chiu:} A bit of a challenge was figuring out just how detailed each test case needed to be. I kept wondering if I was getting too specific or not specific enough. To fix this, I made a quick checklist of essential edge cases based on the project requirements and ran it by the team. Getting their input reassured me that I was covering the right bases without going overboard.

\textbf{Gurnoor Bal:} The main problem was with some of the requirements that were a bit too general or vague. It was hard to nail down exactly what needed verifying. I decided to check in with our project supervisor to get a better handle on what was expected, and I left a few notes in the document to flag areas we might need to clarify later. That way, the process stayed clear without me having to guess on details.

\textbf{Jared Paul:} One thing that wasn’t easy was verifying the interfaces for each module. Some of the connections needed more detail to line up with the rest of the design. To sort this out, I went back and added specific inputs and outputs for each module, which made sure everything matched up well with the Module Guide. It took a bit more time, but in the end, it made the design much clearer.

\textbf{Ahmad Hamadi:} The tricky part was making sure the tests weren’t just covering easy scenarios. I wanted to be sure they were meaningful. I ran into a few issues with environment consistency, which caused random test failures. To fix this, I standardized the setup and added a few retries for tests that occasionally flaked. Then I went back to the requirements to make sure the tests were focused on the important functions. This way, the tests ended up being both reliable and relevant.

\subsection{Question 3}
What knowledge and skills will the team collectively need to acquire to
successfully complete the verification and validation of your project?
Examples of possible knowledge and skills include dynamic testing knowledge,
static testing knowledge, specific tool usage, Valgrind etc.  You should look to
identify at least one item for each team member.\\

Several different and diverse skills are required in order to create the verification and validation plan for this Chest X-ray Diffusion Model Research project.

\textbf{Software Specification Design}\\
Our team will need a strong understanding of software specification design to grasp the project context and outline the required functions and modules. By defining each function’s purpose, inputs, and edge cases, we can plan the implementation of functions accurately according to the specifications. This design phase will serve as the foundation for later development stages.

\textbf{Unit Test Planning}\\
Developing robust unit tests will be crucial. Our team must create comprehensive tests that ensure each function achieves its intended purpose reliably. This will require us to account for all relevant edge cases to avoid misleading results and ensure the accuracy and completeness of our unit tests as a core skill for our V\&V plan.

\textbf{Software Design Principles}\\
To maintain a flexible and maintainable codebase, we’ll need to prioritize software design principles, particularly the SOLID principles. For example, we plan to structure modules to follow the Single Responsibility Principle, which will help ensure clarity and cohesion in the design, allowing future engineers to work with the code more effectively.

\textbf{Diffusion Model Concepts and Implementation}\\
Understanding both the theoretical and practical aspects of diffusion models will be essential for designing an accurate V\&V plan. Our team will need knowledge of the inputs, expected outputs, and common data structures used in similar applications to develop thorough and effective tests.

\subsection{Question 4}
For each of the knowledge areas and skills identified in the previous
question, what are at least two approaches to acquiring the knowledge or
mastering the skill?  Of the identified approaches, which will each team
member pursue, and why did they make this choice?\\

\begin{enumerate}
  \item \textbf{Software Specification Design}\\
    Revise existing papers on software specification design, for example material on MIS specifications, and work done by David Parnas
    Reflect on previous Mcmaster Software Engineering course on software design (i.e. 3A04)

  \item \textbf{Unit Test Planning}\\
    Revise testing blogs produced by large tech companies such as Meta, Google and Amazon. These blogs devise the strategies these companies adopt in their creation of effective unit tests.
    Reflect on previous Mcmaster Software Engineering course, 3S03 Software Testing

  \item \textbf{Software Design Principles}\\
    Revise SOLID principles via informational blogs such as DigitalOcean, Medium, and FreecodeCamp
    Implement personal projects that are small but correctly implement each of the design principles.

  \item \textbf{Diffusion Model Concepts/Implementation}\\
    Utilize Introduction to Diffusion Models course on Udemy. This goes over theory and implementation in practice today.
    Read existing papers on experiments with diffusion models by other researchers that have similar applications
\end{enumerate}

Now, each team member selected a particular method to pursue, and stated their justifications below:

\textbf{Jared Paul:} I’m going to revise more formal and professional software specification design practices by looking at existing papers, such as that from David Parnas. This is because I recall being introduced to him in 2nd year, and found the content really straightforward and informative the, so I thought it would be able to get me up to speed quite quickly today.

\textbf{Hamza Issa:} I chose to revise existing engineering blogs from big tech companies like Amazon, as although there are papers and lots of theory out there for devising unit tests. I think that in designing and creating the best unit tests we can look to companies that are highly dependent on them, to find tests that are not just complete but not overly verbose.

\textbf{Ahmad Hamadi:} I plan on revising my SOLID principles using popular blogs like I’ve seen from FreeCode camp. I think this would be the simplest way to learn all the principles in a manner that is easy to digest and also easy to implement myself for when I want to experiment and practice.

\textbf{Harrison Chiu:} I am going to try and complete the Udemy course on Introduction to Diffusion models. I have some experience looking at research papers for Diffusion models, but they have been really difficult to understand due to the advanced mathematics. I think using a Udemy course would make it simpler to learn and easily transferable to our project

\textbf{Gurnoor Bal:} I am going to investigate existing research papers in the Diffusion Models space that have some similarity to our application. I think this would not only help me learn the required concepts, but also show me how to actually produce research and an application catered for this specific project. If i were to just watch the Udemy course, I’ll understand the concepts, but I would still need to review papers to understand how to go about it for this particular project.

\end{document}

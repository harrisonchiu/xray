\documentclass{article}

\usepackage{booktabs}
\usepackage{tabularx}
\usepackage{hyperref}
\usepackage{parskip}
\usepackage{multirow}
\usepackage{array}
\usepackage{pdflscape}
\usepackage[margin=1.75in]{geometry}

\hypersetup{
  colorlinks=true,       % false: boxed links; true: colored links
  linkcolor=red,          % color of internal links (change box color with linkbordercolor)
  citecolor=green,        % color of links to bibliography
  filecolor=magenta,      % color of file links
  urlcolor=cyan           % color of external links
}

\title{Hazard Analysis\\\progname}

\author{\authname}

\date{}

%% Comments

\usepackage{color}

\newif\ifcomments\commentstrue %displays comments
%\newif\ifcomments\commentsfalse %so that comments do not display

\ifcomments
\newcommand{\authornote}[3]{\textcolor{#1}{[#3 ---#2]}}
\newcommand{\todo}[1]{\textcolor{red}{[TODO: #1]}}
\else
\newcommand{\authornote}[3]{}
\newcommand{\todo}[1]{}
\fi

\newcommand{\wss}[1]{\authornote{blue}{SS}{#1}} 
\newcommand{\plt}[1]{\authornote{magenta}{TPLT}{#1}} %For explanation of the template
\newcommand{\an}[1]{\authornote{cyan}{Author}{#1}}

%% Common Parts

\newcommand{\progname}{Scanalyze AI} % PUT YOUR PROGRAM NAME HERE
\newcommand{\authname}{Team 16, Ace
\\ Hamza Issa
\\ Ahmad Hamadi
\\ Jared Paul
\\ Gurnoor Bal} % AUTHOR NAMES                  

\usepackage{hyperref}
    \hypersetup{colorlinks=true, linkcolor=blue, citecolor=blue, filecolor=blue,
                urlcolor=blue, unicode=false}
    \urlstyle{same}
                                


\begin{document}

\maketitle
\thispagestyle{empty}

~\newpage

\pagenumbering{roman}

\begin{table}[hp]
  \caption{Revision History} \label{TblRevisionHistory}
  \begin{tabularx}{\textwidth}{llX}
    \toprule
    \textbf{Date} & \textbf{Developer(s)} & \textbf{Change}\\
    \midrule
    25 October 2024 & Harrison & Section 5, 6 \\
    25 October 2024 & Hamza & Section 6 \\
    25 October 2024 & Jared & Section 1, 3 \\
    25 October 2024 & Gurnoor & Section 2 \\
    25 October 2024 & Ahmad & Section 7\\
    \bottomrule
  \end{tabularx}
\end{table}

~\newpage

\pagenumbering{arabic}

\section{Introduction}

This document outlines the potential hazards and corresponding controls for the "AI for Chest
X-ray" project, focusing on convolutional neural networks (CNNs) to automatically detect lung and
heart conditions. A hazard, in this context, refers to any condition, event, or malfunction that
could compromise system accuracy, reliability, security, or patient safety. Conducting a thorough
hazard analysis is crucial to mitigate risks, ensure legal compliance (e.g., HIPAA), and improve
the system's safety and reliability for clinical decision support, while safeguarding patient
data integrity and confidentiality.

In developing machine learning systems for medical diagnostics, hazard analysis is essential to
identify potential risks, whether from system failures, unintended consequences, or improper
functionality. These risks can lead to harm, mislead users with erroneous outputs, or negatively
impact the system’s performance. For this project, hazards encompass both software issues and
interactions between the AI system and users, such as radiologists and medical institutions. The
goal of this hazard analysis is to assess these risks, evaluate their potential impacts, and
implement measures to ensure the system functions safely and effectively in a real-world medical
environment.

\section{Scope and Purpose of Hazard Analysis}

The purpose of this hazard analysis is to identify and mitigate risks associated with the "AI for
Chest X-ray" system, particularly in clinical environments. Hazards may arise from various issues,
including misdiagnosis (false positives/negatives), loss of data integrity, unauthorized access
to patient records, and system downtime. The goal is to implement technical, procedural, and
security controls that ensure the system operates safely and reliably while supporting
radiologists in their diagnoses. This analysis covers the entire AI system's lifecycle—from data
acquisition, image analysis, and diagnosis reporting to user interactions and system maintenance.

By focusing on these areas, the hazard analysis aims to safeguard the accuracy of the diagnostic
process, protect patient privacy, and prevent errors that could lead to misdiagnoses or delays in
care. The potential losses associated with these hazards include not only misdiagnosis and delayed
diagnoses but also security breaches and a loss of trust among medical professionals. Given the
high stakes involved in medical decision-making, addressing these hazards comprehensively is
critical to ensuring patient safety and system reliability.

\section{System Boundaries and Components}

The system under analysis is designed to facilitate the detection of lung and heart conditions
from chest X-ray images using advanced technologies. Its key components include the Machine
Learning Model, specifically Convolutional Neural Networks (CNNs), which processes the X-rays to
identify potential health issues. Complementing this is the Diffusion Model, which generates
realistic chest X-ray images that enhance the training of the CNN. The Web Application serves as
the interface for radiologists and researchers, allowing them to upload images and review the
outputs generated by the AI.

Additionally, the system relies on publicly available datasets like CheXpert and MIMIC for
training the AI model, raising concerns about dataset accuracy and bias.

In terms of functionality, the system includes:

\textbf{Frontend}
\begin{enumerate}
  \item A web interface for user interaction and secure login authentication.
  \item Features for image upload, retrieval, and display of diagnostic findings.
  \item Access to diagnostic notes and reports restricted to authorized personnel.
\end{enumerate}

\textbf{Backend}
\begin{enumerate}
  \item An AI model endpoint for detecting anomalies in chest X-rays.
  \item Databases that store patient records, X-ray images, and diagnostic reports.
  \item Connections to external systems, including hospital IT infrastructure and Radiology
    Information Systems (PACS).
\end{enumerate}

The system boundary is defined by these components and their interactions, focusing on the direct
relationship between healthcare professionals and the AI while excluding external factors like
hospital server management and potential network outages that are beyond our control. This
ensures that the hazard analysis remains centered on elements that directly impact the system's
safety, reliability, and effectiveness.

\section{Critical Assumptions}

In conducting the hazard analysis for the CNN-based chest X-ray analysis system, several critical
assumptions are made regarding the functionality, environment, and usage of the system. These
assumptions are necessary to define the scope of potential hazards and their mitigation
strategies. While assumptions help simplify the analysis, they also highlight areas where
potential risks may need to be revisited as the system evolves.

\begin{enumerate}
  \item \textbf{Assumption 1: Model Performance} \\
    It is assumed that the convolutional neural network (CNN) model will perform within the
    expected accuracy levels (≥ 90\%) as defined by the system requirements. While the system is
    tuned for high performance, this assumption leaves open the risk of false positives or false
    negatives in diagnosis, which could lead to misdiagnosis if not managed properly.
  \item \textbf{Assumption 2: Dataset Quality and Relevance} \\
    It is assumed that the datasets used for training (CheXpert, MIMIC) are of high quality and
    representative of real-world chest X-ray images, both in terms of diversity and annotation
    accuracy. This implies that the data contains no significant biases or labeling errors;
    however, if these assumptions are incorrect, the model may generate inaccurate or biased
    results.
  \item \textbf{Assumption 3: Web Application Stability} \\
    The web application is assumed to operate reliably under typical internet conditions, with
    minimal risk of downtime or interruptions. Any disruption in the service (e.g., server
    outages, high latency) could impact user experience and limit accessibility, especially in
    time-sensitive clinical situations.
  \item \textbf{Assumption 4: Data Privacy and Security} \\
    It is assumed that the system does not collect or store any personally identifiable
    information (PII) or sensitive health data from users. This assumption is crucial for
    minimizing privacy risks; however, failure to anonymize or secure data properly could lead to
    breaches, exposing users to legal and ethical issues related to healthcare privacy.
  \item \textbf{Assumption 5: User Competency} \\
    It is assumed that the primary users (radiologists, researchers) possess adequate medical
    knowledge to interpret the system’s outputs. The system does not explain its internal
    decision-making processes ("black box" issue), which could lead to misinterpretation of
    results by inexperienced users. If this assumption is invalid, there is a heightened risk of
    incorrect diagnoses.
  \item \textbf{Assumption 6: Integration with Hospital IT Infrastructure} \\
    The system is assumed to integrate seamlessly with secure hospital IT infrastructure, ensuring
    that patient data (X-rays and records) is stored and accessed securely in compliance with
    HIPAA regulations. This integration is vital for the effective and safe operation of the AI
    system in clinical environments.
  \item \textbf{Assumption 7: Availability of Server Resources} \\
    It is assumed that the server resources, both frontend and backend, will be continuously
    available, except during planned maintenance or unexpected disruptions. Any unanticipated
    downtime could hinder the system's functionality and impact clinical workflows.
\end{enumerate}

\section{Failure Mode and Effect Analysis}

\newgeometry{left=0.25in,right=0.25in,top=0.25in,bottom=0.25in}
\begin{landscape}
  \begin{table}[hp]
    \caption{FMEA Worksheet Part 1} \label{FMEA}
    \centering
    \begin{footnotesize}
      \begin{tabular}{|p{0.8in}|p{1in}|p{1.2in}|p{1.5in}|p{0.6in}|p{0.8in}|p{0.3in}|p{2.2in}|p{0.2in}|p{0.2in}|}
        % Header
        \toprule
        \textbf{Component} & \textbf{Failure Modes} & \textbf{Effects of Failure} &
        \textbf{Causes of Failure} & \textbf{Detection} & \textbf{Controls} & \textbf{Risk} &
        \textbf{Recommended Action} & \textbf{Req.} & \textbf{Ref.} \\
        \bottomrule

        \multirow{2}{1in}{Web Application: User Access Authentication} &
        \multirow{2}{1in}{Fail to authenticate user} &
        Medical professional cannot login &
        \multirow{2}{1.5in}{Authentication error in web application} &
        \multirow{2}{0.6in}{Manual testing} & &
        \multirow{2}{0.3in}{Low} &
        Allow users to have alternative methods to login and be authenticated &
        AR1, SR2 &
        H1.1 \\
        \cline{3-3}\cline{8-10}
        & &
        Unauthorized third party attempting to login & & & & &
        Include security safeguards to prevent unauthorized parties from logging in such as
        passwords, or only making the web application accessible locally. &
        SR2, SR3, AR0, AR2 &
        H1.2 \\
        \hline

        Web Application: Loading Images from Backend &
        Fails to load chest x-ray image &
        Users cannot see chest x-ray images &
        Failed to encode image or failed to send data to frontend from backend &
        Manual testing & &
        Low &
        Include failsafe if backend could not send image to user. And try sending it again if
        possible. & &
        H2 \\
        \hline

        Web Application: Uploading Images to Backend &
        Fails to upload chest x-ray image &
        Backend server which runs the model cannot receive input in order to run disease
        detection &
        Failed to encode image or failed to send data to frontend from backend &
        Manual testing & &
        Low &
        Add redundant data transfer when sending image data. Ensure network connection to backend
        is setup correctly & &
        H3 \\
        \hline

        \multirow{2}{1in}{Web Application: Display Findings} &
        Fails to show diagnostic reports of an x-ray image &
        User cannot read the diagnostic summary report of finding diseases in the images &
        \multirow{2}{1.5in}{Failed to send results from backend or failed to parse output from the
        model into a summary} &
        \multirow{2}{0.6in}{Manual testing} & &
        \multirow{2}{0.3in}{Low} &
        Include other ways for users to read the reports generated from the model & &
        H4.1 \\
        \cline{2-3}\cline{8-10}
        &
        Shows an incorrect disease report &
        User reads a diagnostic report unrelated to the image at hand & & & & &
        Double check if the output of the model is correct &
        SR0 &
        H4.2 \\
        \hline

        \multirow{2}{1in}{Model: Disease detection in the chest x-ray image} &
        False positive detection &
        Healthy patient could be diagnosed; waste of treatment and healthcare time &
        Model is too sensitive to certain patterns and shapes, causing it to detect diseases in
        normal x-ray images &
        \multirow{2}{0.6in}{Automated validation testing during model training} & &
        \multirow{2}{0.3in}{High} &
        Optimize the neural network model to minimize false positives &
        SR0, SR1 &
        H5.1 \\
        \cline{2-4}\cline{8-10}
        &
        False negative detection &
        Patient does not correctly diagnosed and goes untreated; could worsen its state &
        Model is too insensitive to certain shapes and patterns, so it fails to detect disease in
        x-ray images with the diseases & & & &
        Optimize the neural network model to minimize false negatives &
        SR0, SR1 &
        H5.2 \\
        \hline

        Model: Training &
        Model overfitting &
        Model performs very well on training data with high accuracy but has poor accuracy on
        validation data (unseen images). Causes inaccurate detection &
        Overfitting due to high number of training epochs or overly complex model architecture
        layers &
        Validation testing after training &
        Use early stoppage, large training dataset, and regularization &
        High &
        Use techniques mentioned in "Controls" column to detect and prevent overfitting &
        SR0, SR1 &
        H6 \\
        \bottomrule
      \end{tabular}
    \end{footnotesize}
  \end{table}
\end{landscape}
\restoregeometry

\newgeometry{left=0.25in,right=0.25in,top=0.25in,bottom=0.25in}
\begin{landscape}
  \begin{table}[hp]
    \caption{FMEA Worksheet Part 2} \label{FMEA}
    \centering
    \begin{footnotesize}
      \begin{tabular}{|p{1in}|p{1in}|p{1in}|p{1.5in}|p{0.6in}|p{0.8in}|p{0.7in}|p{1.8in}|p{0.2in}|p{0.2in}|}
        % Header
        \toprule
        \textbf{Component} & \textbf{Failure Modes} & \textbf{Effects of Failure} &
        \textbf{Causes of Failure} & \textbf{Detection} & \textbf{Controls} & \textbf{Risk} &
        \textbf{Recommended Action} & \textbf{Req.} & \textbf{Ref.} \\
        \bottomrule

        \multirow{2}{1in}{Data Security} &
        Cyberattacks &
        Unauthorized access database which stores patient health records &
        Weak security controls. This can include unsecured systems and login data unencrypted &
        \multirow{2}{0.6in}{Unauthorized access detection} &
        Good security policies &
        \multirow{2}{0.8in}{Risk of leaking patient health records in a data breach} &
        Improve cybersecurity &
        \multirow{2}{0.2in}{AR0, AR2, SR3} &
        H7.1 \\
        \cline{2-4}\cline{6-6}\cline{8-8}\cline{10-10}
        &
        Unauthorized access &
        Patient privacy at risk and possible data leak &
        Weak access control measures & &
        Minimize unnecessary access with with access groups & &
        Improve access control & &
        H7.2 \\
        \hline

        Access data from Database &
        Data transfer failed &
        Failed to retrieve patient data &
        Network communication issues &
        Automated data transfer checks with network handshakes &
        Redundant data transfer paths &
        Data retrieval risk &
        Add data transfer redundancy when sending data &
        SR3 &
        H8 \\
        \hline

        \multirow{2}{1in}{Backend Server} &
        Network failure &
        Connection to database is disrupted &
        Network connectivity issues &
        \multirow{2}{0.6in}{Monitoring server performance} &
        Redundant network connections &
        \multirow{2}{0.9in}{Operational disruption} &
        Use network connection redundancy & &
        H9.1 \\
        \cline{2-4}\cline{6-6}\cline{8-8}\cline{10-10}
        &
        Server downtime &
        Unable to access database or run model &
        Server overloaded with tasks blocking new tasks & &
        Distributed systems & &
        Use distributed systems to ensure server is always on & &
        H9.2 \\
        \hline

        \multirow{2}{1in}{Data Storage} &
        Data loss &
        Patient images and data is lost &
        Database accidentally deleted data &
        Regular data backups &
        Data redundancy by storing data in multiple places at once &
        Data loss risk &
        Regular backup and have a robust data storage system &
        \multirow{2}{0.2in}{SR3, AR2} &
        H10.1 \\
        \cline{2-8}\cline{10-10}
        &
        Data corruption &
        Patient images and data is lost &
        Database server corruption or data got corrupted during transmission to database &
        Data integrity checks &
        Regular data backups &
        Data loss leak &
        Check integrity of data in database regularly & &
        H10.2 \\
        \bottomrule
      \end{tabular}
    \end{footnotesize}
  \end{table}
\end{landscape}
\restoregeometry

\section{Safety and Security Requirements}

\subsection{Access Requirements}
\textbf{AR0} The x-ray images used as training data should not be made available publicly; rather,
they must be stored on a secure database system to prevent unauthorized access.

Rationale: This is to ensure that private patient data remains secure and is not accessible by
unauthorized individuals or bad actors.

Fit Criterion: Only users with appropriate authorization credentials (e.g. researchers or system
administrators) should be able to access the training data via the secure database.

\textbf{AR1} The website should be made publicly accessible to anyone, allowing users to
experiment with the associated research findings, including generating synthetic chest X-ray
images.

Rationale: This is to allow wide dissemination and experimentation with the research findings,
ensuring that researchers and the public can interact with the model without restrictions.

Fit Criterion: Any user, without needing credentials, should be able to access the website and
experiment with generating synthetic chest X-ray images.

\subsection{Integrity Requirements}
\textbf{AR2} The system will encrypt all chest-x ray training data within the secure database.

Rational: This data should not be available to any website user

Fit criterion: All data is not stored in any discernible language

\subsection{Safety Requirements}
\textbf{SR0} The system will indicate the accuracy of the generated chest X-ray image relative to
real-world test data.

Rationale: Accurate generation is crucial to ensure the validity of research results and their
practical application in the medical field.

Fit Criterion: The system will compare generated images with real test data and display a
precision and accuracy score.

\textbf{SR1} The model will routinely experiment with different training data sets and model
parameters to ensure optimal accuracy and precision of generated results.

Rationale: Continuous experimentation and optimization are needed to refine the diffusion model
and improve the quality of generated data.

Fit Criterion: Results of the experiments will show improvement in precision and accuracy over
time, as tested with various data sets.

\textbf{SR2} There will be no requirement to log in to the platform, as users are not required to
input any personal details. The platform serves purely as an opportunity to experiment with the
findings of the research and model.

Rationale: Removing the need for authentication simplifies access and ensures that no personal
data is collected, enhancing privacy.

Fit Criterion: Users will have unrestricted access to the platform without the need for
authentication.

\textbf{SR3} Patient's data will be encrypted during data transfers.

Rationale: Encryption is needed to secure and ensure the privacy of patient's health data. It also
prevents unauthorized access.

Fit Criterion: The system and database will use a secure encryption algorithm like AES.

\section{Roadmap}

Given the project's goals and requirements, our focus will be on implementing core access,
integrity, and safety requirements. However, due to the research-oriented nature of the project
and limited development time, we will prioritize essential functionalities initially, while some
advanced features and security enhancements will be deferred for future phases.

\textbf{To be implemented during the capstone timeline:}
\begin{itemize}
  \item \textbf{AR0, AR1:} Secure storage of training data and public website access, scheduled for
    implementation after the proof-of-concept demo
  \item \textbf{SR0, SR1:} Displaying accuracy indicators for synthetic chest X-ray images in
    relation to real-world data, scheduled for implementation by January 2025.
\end{itemize}

\textbf{To be implemented in the future:}
\begin{itemize}
  \item \textbf{Algorithm Optimization:} Continuous refinement of model parameters and training
    data to improve precision and realism of synthetic X-ray images.
  \item \textbf{Audit Log Maintenance:} Developing a logging system to track access and
    interactions with the secure database for research tracking purposes.
  \item \textbf{Regular Security Audits:} Periodic reviews and vulnerability assessments to ensure
    data integrity and model accuracy, preventing unauthorized access or misuse of training data.
  \item \textbf{Encryption Updates:} Continuously update encryption standards and protocols to
    align with best practices in data protection and cybersecurity, ensuring ongoing compliance and
    security of stored data.
\end{itemize}

\newpage{}

\section*{Appendix --- Reflection}

The purpose of reflection questions is to give you a chance to assess your own
learning and that of your group as a whole, and to find ways to improve in the
future. Reflection is an important part of the learning process.  Reflection is
also an essential component of a successful software development process.  

Reflections are most interesting and useful when they're honest, even if the
stories they tell are imperfect. You will be marked based on your depth of
thought and analysis, and not based on the content of the reflections
themselves. Thus, for full marks we encourage you to answer openly and honestly
and to avoid simply writing ``what you think the evaluator wants to hear.''

Please answer the following questions.  Some questions can be answered on the
team level, but where appropriate, each team member should write their own
response:


\begin{enumerate}
  \item What went well while writing this deliverable?

    One aspect that went well was defining the safety and security requirements for the project.
    We had already defined non functional requirements in the SRS document, which provided us with
    a clear idea of how to execute this task and format it. Making the assumptions went smoothly
    since we were all on the same page and had a solid understanding of what assumptions to
    make. Our previous work on the project contributed to this clarity, making the process more
    efficient. Team collaboration went smoothly, as we were able to brainstorm effectively and
    divide the tasks efficiently.

  \item What pain points did you experience during this deliverable, and how
    did you resolve them?

    One of the main challenges we faced was determining how to limit the system boundaries and
    what components to include. Given the complexity of our product and its integration
    with external systems (such as hospital IT infrastructure and datasets), it was difficult to
    know which elements to focus on within the scope of the hazard analysis. Another complicating
    factor is that much of our project is research-based, meaning that as we make new discoveries
    or refine our approach, the system boundaries and components may need to be changed. This
    uncertainty made it challenging to confidently establish a stable set of boundaries for the
    analysis. We addressed this by keeping the analysis flexible, acknowledging that components
    and boundaries may shift over time as we progress in our research. This way, we can revisit
    and adjust the hazard analysis as necessary to accommodate any new directions the project may
    take.

  \item Which of your listed risks had your team thought of before this
    deliverable, and which did you think of while doing this deliverable? For
    the latter ones (ones you thought of while doing the Hazard Analysis), how
    did they come about?

    \textbf{Risks Identified Before Deliverable}
    \begin{itemize}
      \item Failure to Authenticate User (Web Application: User Access Authentication)
      \item Failure to Load Chest X-ray Images (Web Application: Loading Images from Backend)
      \item Failure to Display Report of an X-ray Image (Web Application: Display)
      \item False Positive Detection (Model: Disease Detection in Chest X-ray Image)
      \item False Negative Detection (Model: Disease Detection in Chest X-ray Image)
      \item Model Overfitting (Model: Training)
      \item Cyberattacks (Data Security)
      \item Data Transfer Failed(Access Data from Database)
      \item Network Failure (Backend Server)
      \item Server Downtime (Backend Server)
      \item Data Loss (Data Storage: Data Loss)
      \item Data Corruption
    \end{itemize}

    \textbf{Risks Identified During Deliverable}
    \begin{itemize}
      \item Failure to Upload Chest X-ray Image (Web Application: Uploading Images to Backend)

        This risk came about during the hazard analysis, when we conducted an assessment of the 
        data flow from the web application to the backend. This assessment was part of our process 
        for the system boundaries and components section which provided a better understanding of 
        our application. This led us to recognize potential failures in image uploads due to
        connectivity issues or encoding errors. This insight emerged from systematically analyzing
        how user interactions with the frontend could encounter issues during backend integration.

      \item Unauthorized access (Data Security)

        We identified this risk during the safety and security requirements section. Specifically,
        we consider scenarios where data might be breached and accessed by unknown/unauthorized
        members. Training data incorporates real-life patient data which should be protected, and 
        so the risk of cyberattacks is a crucial consideration.

      \item Incorrect Display of Disease Report (Web Application: Display of Findings)

        As part of the hazard analysis, we looked into how diagnostic findings are presented to
        users (User interface characteristics). We identified that incorrect parsing or
        summarization of data could lead to inaccurate outputs on the interface. By examining each
        layer of interaction between the backend processing and frontend display, we discovered 
        that failures in interpreting the model's output could cause misleading reports.
    \end{itemize}

  \item Other than the risk of physical harm (some projects may not have any
    appreciable risks of this form), list at least 2 other types of risk in
    software products. Why are they important to consider?

    \textbf{Security Risks:} Unauthorized access or data breaches could expose sensitive patient
    information. Security risks are critical to consider, especially in healthcare, because they
    can lead to legal liabilities and the patient’s to lose trust in our process.

    \textbf{Reliability Risks:} Software downtime or malfunction, especially in critical
    environments like healthcare, can delay diagnosis, leading to unsatisfied patients and
    negative outcomes. Ensuring high system reliability is essential to maintain efficiency and
    avoid any negative impact on patient care.
\end{enumerate}

\end{document}

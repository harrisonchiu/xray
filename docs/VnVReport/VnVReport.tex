\documentclass[12pt, titlepage]{article}

\usepackage{booktabs}
\usepackage{tabularx}
\usepackage{hyperref}
\usepackage{parskip}
\usepackage{longtable}
\hypersetup{
    colorlinks,
    citecolor=black,
    filecolor=black,
    linkcolor=red,
    urlcolor=blue
}
\usepackage[round]{natbib}

%% Comments

\usepackage{color}

\newif\ifcomments\commentstrue %displays comments
%\newif\ifcomments\commentsfalse %so that comments do not display

\ifcomments
\newcommand{\authornote}[3]{\textcolor{#1}{[#3 ---#2]}}
\newcommand{\todo}[1]{\textcolor{red}{[TODO: #1]}}
\else
\newcommand{\authornote}[3]{}
\newcommand{\todo}[1]{}
\fi

\newcommand{\wss}[1]{\authornote{blue}{SS}{#1}} 
\newcommand{\plt}[1]{\authornote{magenta}{TPLT}{#1}} %For explanation of the template
\newcommand{\an}[1]{\authornote{cyan}{Author}{#1}}

%% Common Parts

\newcommand{\progname}{Scanalyze AI} % PUT YOUR PROGRAM NAME HERE
\newcommand{\authname}{Team 16, Ace
\\ Hamza Issa
\\ Ahmad Hamadi
\\ Jared Paul
\\ Gurnoor Bal} % AUTHOR NAMES                  

\usepackage{hyperref}
    \hypersetup{colorlinks=true, linkcolor=blue, citecolor=blue, filecolor=blue,
                urlcolor=blue, unicode=false}
    \urlstyle{same}
                                


\begin{document}

\title{Verification and Validation Report: \progname} 
\author{\authname}
\date{\today}
	
\maketitle

\pagenumbering{roman}

\section{Revision History}

\begin{tabularx}{\textwidth}{p{3cm}p{2cm}X}
\toprule {\bf Date} & {\bf Version} & {\bf Notes}\\
\midrule
10 March & 1.0 & Gurnoor and Harrison add section 3\\
\bottomrule
\end{tabularx}

~\newpage

\section{Symbols, Abbreviations and Acronyms}

Table 1, includes the definitions and descriptions of all relevant symbols,
abbreviations and acronyms used in this VnV Plan document.

\begin{longtable}[c]{|p{0.3\textwidth}|p{0.7\textwidth}|}
  \hline
  \textbf{Symbol, Abbreviation or Acronym} & \textbf{Definiton or Description} \\ \hline
  \textbf{ML} & Machine Learning: A branch of artificial intelligence that involves the use of algorithms to allow computers to learn from and make predictions based on data. This is a core technology used in the project for analyzing chest X-rays. \\ \hline
  \textbf{DL} & Deep Learning: A subset of machine learning involving neural networks with many layers, used to analyze various types of data, including images. \\ \hline
  \textbf{DICOM} & Digital Imaging and Communications in Medicine: A standard for transmitting, storing, and sharing medical imaging information. It is used to manage medical images in the proposed solution. \\ \hline
  \textbf{CNN} & Convolutional Neural Network: A type of deep learning model specifically designed for processing structured grid data like images, used in the project for chest X-ray analysis. \\ \hline
  \textbf{EHR} & Electronic Health Record: A digital version of a patient's paper chart, used for storing patient information and history that may be integrated with the proposed solution. \\ \hline
  \textbf{API} & Application Programming Interface: A set of rules and protocols for building and interacting with software applications, enabling the integration of the proposed solution with other systems. \\ \hline
  \textbf{MC} & Mandated Constraints: Various constraints placed on the project’s proposed solution that must be adhered to throughout the development process. \\ \hline
  \textbf{FR} & Functional Requirement: A requirement that specifies what functionality the project’s proposed solution must provide to meet user needs. \\ \hline
  \textbf{NFR} & Nonfunctional Requirement: A requirement that specifies criteria that can be used to judge the operation of a system, rather than specific behaviors (e.g., performance, usability). \\ \hline
  \textbf{BUC} & Business Use Case: A scenario that describes how the proposed solution can be used within a business context to achieve specific goals. \\ \hline
  \textbf{PUC} & Product Use Case: A scenario that details how an individual user will interact with the proposed solution to achieve specific tasks. \\ \hline
  \textbf{MVP} & Minimum Viable Product: A version of the proposed solution that includes only the essential features required to meet the core needs of the users and stakeholders. \\ \hline
  \textbf{MG} & Module Guide \\ \hline
  \textbf{MIS} & Module Interface Specification \\ \hline
  \textbf{PoC} & Proof of Concept \\ \hline
  \textbf{SRS} & Software Requirements Specification \\ \hline
  \textbf{FRTC} & Functional Requirements Test Case \\ \hline
  \textbf{NFRTC} & Nonfunctional Requirements Test Case \\ \hline
  \textbf{VnV} & Verification and Validation \\ \hline
\end{longtable}

\newpage

\tableofcontents

\listoftables %if appropriate

\listoffigures %if appropriate

\newpage

\pagenumbering{arabic}

This document ...

\section{Functional Requirements Evaluation}

\subsection{User Authentication (FR 8)}
\subsubsection{User Authentication (FRTC15)}
\textbf{Initial State:} The system is operational with user accounts created.\\
\textbf{Input:} Valid login credentials for an authorized user.\\
\textbf{Expected Output:} The system authenticates the user and grants access according to the user's role.\\
\textbf{Actual Output:} The system successfully authenticates the user and redirects them to the dashboard, allowing appropriate access based on their role.\\
\textbf{Expected and Actual Output Match:} True\\
\textbf{Relevant Functional Requirement(s):} FR.8 (Authentication and Authorization Mechanisms)\\

\subsection{Image Upload (FR.1)}
\subsubsection{Chest X-ray Image Input Acceptance (FRTC1)}
\textbf{Initial State:} The system is in a stable state with all components initialized and ready to receive input.\\
\textbf{Input:} A sample chest X-ray image in a valid format (JPG, PNG, DICOM).\\
\textbf{Expected Output:} The system accepts and reads the chest X-ray image successfully.
No error messages or system anomalies occur.\\
\textbf{Actual Output:} The system successfully accepts, reads, and stores the uploaded image in the backend.\\
\textbf{Expected and Actual Output Match:} True\\
\textbf{Relevant Functional Requirement(s):} FR.1 (Image Upload)\\

Invalid Chest X-ray Image Format Rejection (FRTC2)
\textbf{Initial State:} The system is ready to receive an image file.\\
\textbf{Input:} A sample image in an invalid format (e.g., .TXT, .DOCX).\\
\textbf{Expected Output:} The system rejects the invalid image input.\\
An appropriate error message is displayed, informing the user about supported formats.
\textbf{Actual Output:} The system correctly rejects invalid file formats and provides an error message.\\
\textbf{Expected and Actual Output Match:} True\\

\subsection{Image Preprocessing (FR.2)}
\subsubsection{Image Preprocessing and Standardization (FRTC3)}
\textbf{Initial State:} A chest X-ray image has been uploaded and is ready for preprocessing.\\
\textbf{Input:} A valid chest X-ray image in JPG, PNG, or DICOM format.\\
\textbf{Expected Output:} The system resizes the image to match the CNN model’s required dimensions.\\
Pixel intensity values are normalized to enhance consistency.
The preprocessed image is saved and ready for inference.
\textbf{Actual Output:} The system successfully resizes and normalizes images, preparing them for model analysis.\\
\textbf{Expected and Actual Output Match:} True\\
\textbf{Relevant Functional Requirement(s):} FR.2 (Image Preprocessing)\\

\subsubsection{Handling Invalid Image Formats (FRTC4)}
\textbf{Initial State:} The system is awaiting an image upload.\\
\textbf{Input:} An invalid image file (e.g., TXT, DOCX, unsupported formats, or corrupted image files).\\
\textbf{Expected Output:}
\begin{itemize}
\item The system detects that the image is in an unsupported format or corrupted.
\item The system displays an error message to the user.
\item The system prevents further processing and does not send the image to the model.
\end{itemize}
\textbf{Actual Output:} The system successfully identifies invalid formats and blocks the upload, displaying an appropriate error message.\\
\textbf{Expected and Actual Output Match:} True\\
\textbf{Relevant Functional Requirement(s):} FR.2 (Image Preprocessing)\\

\subsection{CNN Model Accurate Analysis (FR.3)}
\subsubsection{Disease Classification with Confidence Scores (FRTC5 - Updated)}
\textbf{Initial State:} The system has preprocessed an uploaded image and is ready for analysis.\\
\textbf{Input:} A preprocessed chest X-ray image with known disease patterns (e.g., Pneumonia, Cardiomegaly, Atelectasis).\\
\textbf{Expected Output:}
\begin{itemize}
\item The CNN model classifies the image and assigns a disease label.
\item The model outputs confidence scores (e.g., Pneumonia: 85\%, No Finding: 10\%).
\item The prediction results are displayed to the user.
\end{itemize}
\textbf{Actual Output:} The CNN model successfully classifies diseases and assigns confidence scores.\\
\textbf{Expected and Actual Output Match:} True\\
\textbf{Relevant Functional Requirement(s):} FR.3 (CNN Model Accurate Analysis)\\

\subsubsection{Model Prediction for No Disease Cases (FRTC6 - Updated)}
\textbf{Initial State:} The CNN model is ready for analysis, and a healthy chest X-ray is uploaded.\\
\textbf{Input:} A preprocessed chest X-ray image of a patient with no known disease.\\
\textbf{Expected Output:}
\begin{itemize}
\item The CNN model classifies the image as No Finding or returns very low probabilities for disease labels.
\item The confidence scores reflect minimal probability of disease.
\item The prediction results are displayed to the user.
\end{itemize}
\textbf{Actual Output:} The system correctly identifies images without disease and returns a No Finding classification.\\
\textbf{Expected and Actual Output Match:} True\\
\textbf{Relevant Functional Requirement(s):} FR.3 (CNN Model Accurate Analysis)\\

\subsection{Display Results (FR.6)}
\subsubsection{Diagnostic Report and Heatmap Access via Web Interface (FRTC10 - Updated)}
\textbf{Input:} The user navigates to the results page after an image has been analyzed.\\
\textbf{Expected Output:} The system displays the predicted disease classification(s). The confidence scores are presented next to each disease label. (Optional) A heatmap overlay is shown, visually indicating affected areas in the X-ray. The interface is formatted clearly for easy interpretation.\\
\textbf{Actual Output:} The system correctly displays predicted disease labels, corresponding confidence scores, and a heatmap visualization (if enabled).\\
\textbf{Expected and Actual Output Match:} True\\
\textbf{Relevant Functional Requirement(s):} FR.6 (Display Results)\\

\subsubsection{Display Prediction Results in Readable Format (FRTC19 - New)}
\textbf{Initial State:} The system has completed model inference, and the user is accessing the results page.\\
\textbf{Input:} The user opens the result page after an image has been analyzed.\\
\textbf{Expected Output:} The predicted disease label(s) are prominently displayed. The corresponding confidence scores are formatted for clarity. No extraneous or misleading information is presented. Results are accessible within a reasonable timeframe (<5 seconds).\\
\textbf{Actual Output:} The system successfully displays predictions in a structured and readable format.\\
\textbf{Expected and Actual Output Match:} True\\
\textbf{Relevant Functional Requirement(s):} FR.6 (Display Results)\\

\subsection{Heatmap Report (FR.7)}
\subsubsection{Heatmap Display on Chest X-ray Images (FRTC9 - Updated)}
\textbf{Initial State:} The CNN model has completed image analysis, and the system has generated a heatmap.\\
\textbf{Input:} The user navigates to the results page and requests a heatmap visualization.\\
\textbf{Expected Output:} The system overlays a heatmap on the chest X-ray image. The heatmap highlights areas where the model detected abnormalities. The overlay does not obscure critical image details.\\
\textbf{Actual Output:} The system successfully generates and overlays the heatmap on the X-ray image, maintaining image clarity.\\
\textbf{Expected and Actual Output Match:} True\\
\textbf{Relevant Functional Requirement(s):} FR.7 (Heatmap Report)\\

\subsubsection{Heatmap Generation Error Handling (FRTC20 - New)}
\textbf{Initial State:} The CNN model has completed analysis, but an error occurs in heatmap generation.\\
\textbf{Input:} The system attempts to generate a heatmap but encounters an issue (e.g., missing heatmap data, computation error).\\
\textbf{Expected Output:} The system displays a message indicating that the heatmap could not be generated. The rest of the diagnostic results (predictions and confidence scores) remain accessible.\\
\textbf{Actual Output:} The system successfully identifies heatmap generation errors and informs the user without affecting other results.\\
\textbf{Expected and Actual Output Match:} True\\
\textbf{Relevant Functional Requirement(s):} FR.7 (Heatmap Report)\\

\subsection{User Dashboard (FR.9)}
\subsubsection{Display User Dashboard with Past Uploads (FRTC21)}
\textbf{Initial State:} The user is logged in and has previously uploaded X-ray images.\\
\textbf{Input:} The user navigates to the dashboard page.\\
\textbf{Expected Output:} The system retrieves and displays a list of past uploaded X-ray images. Each entry includes the upload timestamp and corresponding diagnosis results.\\
\textbf{Actual Output:} The system successfully loads and displays past uploads with timestamps and results.\\
\textbf{Expected and Actual Output Match:} True\\
\textbf{Relevant Functional Requirement(s):} FR.9 (User Dashboard)\\

\subsection{Secure API for Model Inference (FR.9)}
\subsubsection{API Endpoint for Image Inference (FRTC22)}
\textbf{Initial State:} No request is made, and the API is available.\\
\textbf{Input:} A POST request containing a valid chest X-ray image.\\
\textbf{Expected Output:}
\begin{itemize}
\item The API processes the image and forwards it to the CNN model.
\item The API returns a JSON response with the predicted disease label and confidence scores.
\item The response follows the correct structure (e.g., { "disease": "Pneumonia", "confidence": 85\% }).
\end{itemize}
\textbf{Actual Output:} The API successfully processes the request and returns the correct JSON response format.\\
\textbf{Expected and Actual Output Match:} True\\
\textbf{Relevant Functional Requirement(s):} FR.9 (Secure API for Model Inference)\\

\subsubsection{API Handling of Invalid Image Format (FRTC23)}
\textbf{Initial State:} No request is made, and the API is available.\\
\textbf{Input:} A POST request containing an invalid file format (e.g., .TXT, .DOCX).\\
\textbf{Expected Output:}
\begin{itemize}
\item The API rejects the request.
\item The API returns a structured error message (e.g., { "error": "Invalid file format. Please upload \item JPG, PNG, or DICOM." }).
\end{itemize}
\textbf{Actual Output:} The API correctly rejects invalid file formats and returns an appropriate error message.\\
\textbf{Expected and Actual Output Match:} True\\
\textbf{Relevant Functional Requirement(s):} FR.9 (Secure API for Model Inference)\\

\subsection{Data Storage \& Management (FR.7)}
\subsubsection{Secure Image and Result Storage (FRTC22 - Updated)}
\textbf{Initial State:} A user has uploaded an X-ray image, and the system is ready to store the data.\\
\textbf{Input:} A valid chest X-ray image is uploaded, and the CNN model returns predictions.\\
\textbf{Expected Output:}
\begin{itemize}
\item The system stores the uploaded image in a secure storage location.
\item The system stores prediction results in the database, linking them to the respective image.
\item No unauthorized access to stored data occurs.
\end{itemize}
\textbf{Actual Output:} The system successfully stores images and results securely, ensuring accessibility for future retrieval.\\
\textbf{Expected and Actual Output Match:} True\\
\textbf{Relevant Functional Requirement(s):} FR.7 (Data Storage \& Management)

\subsubsection{Retrieval of Stored Results (FRTC23 - Updated)}
\textbf{Initial State:} The database contains stored images and results for a logged-in user.\\
\textbf{Input:} A user navigates to the dashboard and requests to view past results.\\
\textbf{Expected Output:}
\begin{itemize}
\item The system retrieves and displays stored images with associated predictions.
\item Each entry includes timestamps and confidence scores for past analyses.
\item The data retrieval process is efficient, ensuring a smooth user experience.
\end{itemize}
\textbf{Actual Output:} The system successfully retrieves and displays stored images and results on the dashboard.\\
\textbf{Expected and Actual Output Match:} True\\
\textbf{Relevant Functional Requirement(s):} FR.7 (Data Storage \& Management)

\subsection{Error Handling \& Notifications (FR.9)}
\subsubsection{Invalid Image Format Handling (FRTC24 - Updated)}
\textbf{Initial State:} The system is awaiting an image upload.\\
\textbf{Input:} A user attempts to upload an unsupported file format (e.g., TXT, DOCX, MP4).\\
\textbf{Expected Output:}
\begin{itemize}
\item The system rejects the upload.
\item An error message appears stating, “Invalid file format. Please upload a JPG, PNG, or DICOM file.”
\item The user is prompted to try again with a valid file format.
\end{itemize}
\textbf{Actual Output:} The system correctly rejects invalid file formats and displays an appropriate error message.\\
\textbf{Expected and Actual Output Match:} True\\
\textbf{Relevant Functional Requirement(s):} FR.9 (Error Handling \& Notifications)

\subsubsection{Corrupted Image Upload Handling (FRTC25 - Updated)}
\textbf{Initial State:} The system is awaiting an image upload.\\
\textbf{Input:} A user attempts to upload a corrupted chest X-ray image.\\
\textbf{Expected Output:}
\begin{itemize}
\item The system detects that the file is corrupted and cannot be processed.
\item An error message appears stating, “File appears to be corrupted. Please upload a valid image.”
\item The system does not store or process the corrupted file.
\end{itemize}
\textbf{Actual Output:} The system successfully detects and rejects corrupted image files, displaying an appropriate error message.\\
\textbf{Expected and Actual Output Match:} True\\
\textbf{Relevant Functional Requirement(s):} FR.9 (Error Handling \& Notifications)

\subsection{Multi-Disease Classification (FR.11)}
\subsubsection{Multi-Disease Detection with Confidence Scores (FRTC26 - Updated)}
\textbf{Initial State:} The system has preprocessed an uploaded image and is ready for analysis.\\
\textbf{Input:} A preprocessed chest X-ray image containing multiple known diseases (e.g., Pneumonia and Tuberculosis).\\
\textbf{Expected Output:}
\begin{itemize}
\item The CNN model classifies the image and returns a list of detected diseases.
\item The confidence scores for each predicted disease are displayed.
\\textbf{item Example output:} Pneumonia: 85\%, Tuberculosis: 70\%.
\end{itemize}
\textbf{Actual Output:} The CNN model successfully detects multiple diseases in a single image and provides confidence scores.\\
\textbf{Expected and Actual Output Match:} True\\
\textbf{Relevant Functional Requirement(s):} FR.11 (Multi-Disease Classification)\\

\subsubsection{No Multi-Disease Detection for Single Condition (FRTC27 - Updated)}
\textbf{Initial State:} The system has preprocessed an uploaded image and is ready for analysis.\\
\textbf{Input:} A preprocessed chest X-ray image with only one known disease (e.g., Cardiomegaly).\\
\textbf{Expected Output:}
\begin{itemize}
\item The CNN model classifies the image and returns a single disease label.
\item Confidence scores for unrelated diseases remain low.
\textbf{Example output:} Cardiomegaly: 92\%, No Finding: 8\%.
\end{itemize}
\textbf{Actual Output:} The system correctly identifies only one disease when multiple diseases are not present.\\
\textbf{Expected and Actual Output Match:} True\\
\textbf{Relevant Functional Requirement(s):} FR.11 (Multi-Disease Classification)\\

\subsection{Model Confidence (FR.12)}
\subsubsection{Low Confidence Warning for Uncertain Predictions (FRTC28 - Updated)}
\textbf{Initial State:} The system has preprocessed an uploaded image and is ready for analysis.\\
\textbf{Input:} A preprocessed chest X-ray image that leads to low confidence predictions (e.g., all detected diseases have confidence scores below 50\%).
\textbf{Expected Output:}
\begin{itemize}
\item The system returns disease classifications with confidence scores.
\item If the highest confidence score is below 50\%, the system displays a warning: “Low confidence in prediction – consider consulting a radiologist.”
\end{itemize}
\textbf{Actual Output:} The system correctly identifies low-confidence cases and displays the warning message.\\
\textbf{Expected and Actual Output Match:} True\\
\textbf{Relevant Functional Requirement(s):} FR.12 (Model Confidence)\\

\subsubsection{High Confidence Predictions Displayed Normally (FRTC29 - Updated)}
\textbf{Initial State:} The system has preprocessed an uploaded image and is ready for analysis.\\
\textbf{Input:} A preprocessed chest X-ray image that leads to high-confidence predictions (e.g., Pneumonia: 85\%).
\textbf{Expected Output:}
\begin{itemize}
\item The system returns disease classifications with confidence scores.
\item If confidence scores are above 50\%, no warning is displayed.
\end{itemize}
\textbf{Actual Output:} The system correctly displays high-confidence predictions without unnecessary warnings.\\
\textbf{Expected and Actual Output Match:} True\\
\textbf{Relevant Functional Requirement(s):} FR.12 (Model Confidence)\\


\section{Nonfunctional Requirements Evaluation}

\subsection{Usability}
		
\subsection{Performance}

\subsection{etc.}
	
\section{Comparison to Existing Implementation}	

This section will not be appropriate for every project.

\section{Unit Testing}

\section{Changes Due to Testing}

\wss{This section should highlight how feedback from the users and from 
the supervisor (when one exists) shaped the final product.  In particular 
the feedback from the Rev 0 demo to the supervisor (or to potential users) 
should be highlighted.}

\section{Automated Testing}
		
\section{Trace to Requirements}
		
\section{Trace to Modules}		

\section{Code Coverage Metrics}

\bibliographystyle{plainnat}
\bibliography{../../refs/References}

\newpage{}
\section*{Appendix --- Reflection}

The information in this section will be used to evaluate the team members on the
graduate attribute of Reflection.

The purpose of reflection questions is to give you a chance to assess your own
learning and that of your group as a whole, and to find ways to improve in the
future. Reflection is an important part of the learning process.  Reflection is
also an essential component of a successful software development process.  

Reflections are most interesting and useful when they're honest, even if the
stories they tell are imperfect. You will be marked based on your depth of
thought and analysis, and not based on the content of the reflections
themselves. Thus, for full marks we encourage you to answer openly and honestly
and to avoid simply writing ``what you think the evaluator wants to hear.''

Please answer the following questions.  Some questions can be answered on the
team level, but where appropriate, each team member should write their own
response:


\begin{enumerate}
  \item What went well while writing this deliverable? 
  \item What pain points did you experience during this deliverable, and how
    did you resolve them?
  \item Which parts of this document stemmed from speaking to your client(s) or
  a proxy (e.g. your peers)? Which ones were not, and why?
  \item In what ways was the Verification and Validation (VnV) Plan different
  from the activities that were actually conducted for VnV?  If there were
  differences, what changes required the modification in the plan?  Why did
  these changes occur?  Would you be able to anticipate these changes in future
  projects?  If there weren't any differences, how was your team able to clearly
  predict a feasible amount of effort and the right tasks needed to build the
  evidence that demonstrates the required quality?  (It is expected that most
  teams will have had to deviate from their original VnV Plan.)
\end{enumerate}

\end{document}
\documentclass{article}

\usepackage{float}
\restylefloat{table}

\usepackage{booktabs}

\title{Team Contributions: Rev 0\\\progname}

\author{\authname}

\date{}

%% Comments

\usepackage{color}

\newif\ifcomments\commentstrue %displays comments
%\newif\ifcomments\commentsfalse %so that comments do not display

\ifcomments
\newcommand{\authornote}[3]{\textcolor{#1}{[#3 ---#2]}}
\newcommand{\todo}[1]{\textcolor{red}{[TODO: #1]}}
\else
\newcommand{\authornote}[3]{}
\newcommand{\todo}[1]{}
\fi

\newcommand{\wss}[1]{\authornote{blue}{SS}{#1}} 
\newcommand{\plt}[1]{\authornote{magenta}{TPLT}{#1}} %For explanation of the template
\newcommand{\an}[1]{\authornote{cyan}{Author}{#1}}

%% Common Parts

\newcommand{\progname}{Scanalyze AI} % PUT YOUR PROGRAM NAME HERE
\newcommand{\authname}{Team 16, Ace
\\ Hamza Issa
\\ Ahmad Hamadi
\\ Jared Paul
\\ Gurnoor Bal} % AUTHOR NAMES                  

\usepackage{hyperref}
    \hypersetup{colorlinks=true, linkcolor=blue, citecolor=blue, filecolor=blue,
                urlcolor=blue, unicode=false}
    \urlstyle{same}
                                


\begin{document}

\maketitle

This document summarizes the contributions of each team member for the Rev 0
Demo.  The time period of interest is the time between the POC demo and the Rev
0 demo.

\section{Demo Plans}

We will be demoing a full stack machine learning website designed to predict labels for chest x-ray images. The system uses a convolutional neural network model that processes input images and outputs one or more of 16 possible classifications. The React-based front end allows users to upload chest x-ray images, while the backend handles model inference and returns the predicted labels. The demo will showcase the functionality of both the website and the model, followed by a discussion of the challenges we encountered, including model accuracy, data processing, and training time. We will conclude with our planned improvements in model architecture, system performance, user interface, and result verification.

\section{Team Meeting Attendance}

\wss{For each team member how many team meetings have they attended over the
time period of interest.  This number should be determined from the meeting
issues in the team's repo.  The first entry in the table should be the total
number of team meetings held by the team.}

\begin{table}[H]
\centering
\begin{tabular}{ll}
\toprule
\textbf{Student} & \textbf{Meetings}\\
\midrule
Total & 9\\
Hamza Issa & 7\\
Gurnoor Bal & 9\\
Ahmad Hamadi & 9\\
Jared Paul & 8\\
Harrison Chiu & 6\\
\bottomrule
\end{tabular}
\end{table}

Some meetings only involved working on specific parts of the project which are assigned to team members. So, not all meetings are necessary.

\section{Supervisor/Stakeholder Meeting Attendance}

\wss{For each team member how many supervisor/stakeholder team meetings have
they attended over the time period of interest.  This number should be determined
from the supervisor meeting issues in the team's repo.  The first entry in the
table should be the total number of supervisor and team meetings held by the
team.  If there is no supervisor, there will usually be meetings with
stakeholders (potential users) that can serve a similar purpose.}

\begin{table}[H]
\centering
\begin{tabular}{ll}
\toprule
\textbf{Student} & \textbf{Meetings}\\
\midrule
Total & 2\\
Hamza Issa & 2\\
Gurnoor Bal & 2\\
Ahmad Hamadi & 2\\
Jared Paul & 2\\
Harrison Chiu & 2\\
\bottomrule
\end{tabular}
\end{table}

\section{Lecture Attendance}

\wss{For each team member how many lectures have they attended over the time
period of interest.  This number should be determined from the lecture issues in
the team's repo.  The first entry in the table should be the total number of
lectures since the beginning of the term.}

\begin{table}[H]
\centering
\begin{tabular}{ll}
\toprule
\textbf{Student} & \textbf{Lectures}\\
\midrule
Total & 8\\
Hamza Issa & 7\\
Gurnoor Bal & 7\\
Ahmad Hamadi & 8\\
Jared Paul & 8\\
Harrison Chiu & 7\\
\bottomrule
\end{tabular}
\end{table}

\section{TA Document Discussion Attendance}

\wss{For each team member how many of the informal document discussion meetings
with the TA were attended over the time period of interest.}

\begin{table}[H]
\centering
\begin{tabular}{ll}
\toprule
\textbf{Student} & \textbf{Lectures}\\
\midrule
Total & 2\\
Hamza Issa & 2\\
Gurnoor Bal & 2\\
Ahmad Hamadi & 2\\
Jared Paul & 2\\
Harrison Chiu & 2\\
\bottomrule
\end{tabular}
\end{table}

\section{Commits}

\wss{For each team member how many commits to the main branch have been made
over the time period of interest.  The total is the total number of commits for
the entire team since the beginning of the term.  The percentage is the
percentage of the total commits made by each team member.}

\begin{table}[H]
\centering
\begin{tabular}{lll}
\toprule
\textbf{Student} & \textbf{Commits} & \textbf{Percent}\\
\midrule
Total & 7 & 100\% \\
Hamza Issa & 5 & 71\%\\
Gurnoor Bal & 6 & 86\%\\
Ahmad Hamadi & 6 & 86\%\\
Jared Paul & 6 & 86\%\\
Harrison Chiu & 6 & 86\%\\
\bottomrule
\end{tabular}
\end{table}

Most of the commits were authored by multiple people. Looking at the commits for contributions is misleading because some of the commits are very large while some are very small. For some of the commits for MG and MIS, we forgot to add co-authors, so it looks like only 1 person contributed, but everyone worked on it on a separate Google Docs. We counted commit contributions as if we had added those co-authors.

All of the backend and frontend work, mostly by Hamza has not yet been commited. Furthermore, there are 2 commits (Add Hazard Analysis doc and Finish VnVPlan document) which were authored from before the time period of interest. They were recently added because of some issues with merging into main branch and the commits were deleted. Those commits were cherrypicked and added back. These commits were not counted.

\section{Issue Tracker}

\wss{For each team member how many issues have they authored (including open and
closed issues (O+C)) and how many have they been assigned (only counting closed
issues (C only)) over the time period of interest.}

\begin{table}[H]
\centering
\begin{tabular}{lll}
\toprule
\textbf{Student} & \textbf{Authored (O+C)} & \textbf{Assigned (C only)}\\
\midrule
Hamza Issa & 3 & 3\\
Gurnoor Bal & 3 & 3\\
Ahmad Hamadi & 3 & 3\\
Jared Paul & 2 & 2\\
Harrison Chiu & 3 & 3\\
\bottomrule
\end{tabular}
\end{table}

All assigned issues are closed.

\section{CICD}

For now, our main work involves Python. We use it to preprocess our training data and to train our model. Our CI involves a Github Action check on Python code that checks for linting and checks if it is formatted in Black style. Black is a formatting standard in which our team agreed to use. It is a command line tool formatter that should be ran before commiting any Python code. 

We have not yet committed any web development code (JavaScript). This is because we are still in the process of organizing the frontend and backend components. The integration is a bit more complex than expected, and we want to ensure everything is structured properly before pushing to the repo. This will help avoid confusion later and make the codebase easier to maintain. As a result, CI for JavaScript is not yet set up.

\end{document}
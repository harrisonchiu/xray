\documentclass{article}

\usepackage{tabularx}
\usepackage{booktabs}

\title{Problem Statement and Goals\\\progname}

\author{\authname}

\date{}

\input{../Comments}
%% Common Parts

\newcommand{\progname}{Chest Scan} % PUT YOUR PROGRAM NAME HERE
\newcommand{\authname}{Team 16, Ace
\\ Harrison Chiu
\\ Hamza Issa
\\ Ahmad Hamadi
\\ Jared Paul
\\ Gurnoor Bal} % AUTHOR NAMES                  

\usepackage{hyperref}
    \hypersetup{colorlinks=true, linkcolor=blue, citecolor=blue, filecolor=blue,
                urlcolor=blue, unicode=false}
    \urlstyle{same}
                                


\begin{document}

\maketitle

\begin{table}[hp]
\caption{Revision History} \label{TblRevisionHistory}
\begin{tabularx}{\textwidth}{llX}
\toprule
\textbf{Date} & \textbf{Developer(s)} & \textbf{Change}\\
\midrule
21 Sept 2024 & Harrison Chiu & Started basic info of doc \\
24 Sept 2024 & Jared Paul & asdasd \\
\bottomrule
\end{tabularx}
\end{table}

\section{Problem Statement}

This section describes the problem that this project aims to solve. It goes over the general
problem, expected inputs and outputs of the project, the environment, and the stakeholders.


\subsection{Problem}

Chest X-rays are one of the most common diagnostic tools for detecting lung and heart conditions.
However, interpreting them accurately requires expertise. Radiologists must examine each x-ray 
carefully. Sometimes they can miss subtle indicators of a disease. Since chest x-rays are taken
for many different reasons, such discovering lung and cardiac conditions or ruling out diseases, 
there is an increasing volume of medical imaging data. This rising number of x-rays leads to
significant delays in diagnosis, overwhelming radiologists which could cause them to make 
mistakes. As a result, the demand for more efficient and reliable diagnostic support grows. A 
neural network designed to process chest x-rays can help radiologists detect conditions, reduce 
diagnostic errors, and provide faster assessments.

\subsection{Inputs and Outputs}

The high-level \textbf{inputs and outputs} of this problem are described as follows:\\
\textbf{Inputs}:
\begin{itemize}
\item Chest X-ray images sourced from publicly available datasets (i.e. ChexPert, MIMIC) for model training and testing.
\item Existing academic literature on AI improvements to chest x-ray analysis and results accuracy.
\end{itemize}

\noindent \textbf{Outputs}:
\begin{itemize}
\item Tuned convolutional neural network model that attempts to accurately classify particular chest related defects from X-rays, such as cardiac or lung disease.
\item Research paper discussing experiment approach, novel findings and measurable success metric such as precision, and accuracy.
\item Larger web application that enables interested researchers to experiment with our CNN model online.

\end{itemize} 

\subsection{Stakeholders}

The various \textbf{stakeholders} of this problem and this project's proposed solution are described in detail below:

\begin{itemize}
\begin{item}
    \textbf{Radiologists}: Healthcare professionals who will use the AI system for diagnostic assistance. 
    These professionals will need to learn how to use the system, and learn about the practicality of the research in regards to their professional needs.
\end{item}
\begin{item}
    \textbf{Biomedical Researchers}: Researchers that have an interest in the intersection of AI and Healthcare, who will benefit from new research insights and potential conducting their own research in reference to ours.
\end{item}
\begin{item}
    \textbf{Patients}: Individuals whose chest X-rays will be analyzed by the system. Patients will potentially receive a more hassle free service with accurate results as a result of this research.
\end{item}
\begin{item}
    \textbf{AI Researchers and Engineers}: Individuals who are actively working in the AI space who can find practicality in the Chest-X ray CNN research as it applies to their particular field of interest.
\end{item}
\begin{item}
    \textbf{Hospitals and Medical Centers}: Institutions that will implement this technology for faster and more accurate diagnoses. May need to conduct legal liability assessment based on the degree of accuracy of the research results.
\end{item}
\end{itemize}


\subsection{Environment}

The \textbf{environment} of this problem and project's proposed solution are described below:

\begin{itemize}
\begin{item}
\textbf{Software}:  Machine learning frameworks such as TensorFlow or PyTorch for model development, and natural language processing libraries for report generation.
\end{item}
\begin{item}
\textbf{Hardware}: High-performance GPUs or cloud-based systems for model training and deployment.
\end{item}
\end{itemize}

\section{Goals}

\section{Stretch Goals}
The \textbf{stretch goals} of this project's proposed solution are described in detail below:

\begin{itemize}
\begin{item}
    \textbf{Advanced Model Exploration}: Experiment with advanced techniques, such as fine-tuning transformer-based models for medical imaging.
\end{item}
\begin{item} 
    \textbf{Dataset Augmentation}: Investigate methods for augmenting the training dataset to improve model performance on underrepresented conditions. This may include elements of transfer learning techniques, depending on the availability of relevant datasets.
\end{item}
\begin{item} 
    \textbf{Free-Form Report Generation}:Extend the radiology report generation beyond structured outputs to free-form, natural language-based reporting.
\end{item}
\end{itemize}

\section{Challenge Level and Extras}
This project falls into the research-focused, advanced challenge level. The complexity arises from the need to thoroughly review and synthesize state-of-the-art AI techniques in medical imaging while also building an MVP that translates research findings into a functional system. The deliverables emphasize academic inquiry, but with practical application as a secondary goal.\\

\textbf{Extras}:
\begin{itemize}
\begin{item}
    Research Paper Synthesis: Develop a report summarizing key findings from the literature and how they inform the product’s design.
\end{item}
\begin{item}
Usability Testing (Optional): If the MVP is developed to a sufficient level, testing with radiologists may be explored.
\end{item}
\begin{item}
Documentation: Create both research and user-facing documentation that outlines the methodologies used, as well as any findings on model performance.
\end{item}
\end{itemize}

\newpage{}

\section*{Appendix --- Reflection}

\begin{enumerate}
    \item What went well while writing this deliverable? 
    \item What pain points did you experience during this deliverable, and how
    did you resolve them?
    \item How did you and your team adjust the scope of your goals to ensure
    they are suitable for a Capstone project (not overly ambitious but also of
    appropriate complexity for a senior design project)?
\end{enumerate}  

\end{document}